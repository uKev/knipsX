% Für die Description muss man den Satz nicht mit einem Punkt . abschließen, sonst werden unter Umständen zwei Punkte .. im Glossar angezeigt.

\newglossaryentry{kit}{name=KIT,description={Karlsruher Institut für Technologie ist die neue Bezeichnung der früheren Universität Karlsruhe (TH), die seit dem 01.10.2009 ihre Gültigkeit hat},first={Karlsruher Institut für Technologie (vormals Universität Karlsruhe (TH))}}

\newglossaryentry{ipd}{name=IPD Tichy,description={Institut für Programmstrukturen und Datenorganisation am Karlsruher Institut für Technologie, Lehrstuhl Professor Tichy},first={Institut für Programmstrukturen und Datenorganisation am Karlsruher Institut für Technologie, Lehrstuhl Professor Tichy}}

\newglossaryentry{ps}{name=picture set,description={Ist eine Bildmenge, die Bilder, Verzeichnisse und/oder weitere Bildmengen enthalten kann},first={picture set (dt: Bildmenge)}}

\newglossaryentry{rp}{name=Report,description={Ist eine Auswertung, die Bildmengen mit Diagrammtypen verknüpft},first={Report (dt: Auswertung)}}

\newglossaryentry{tempX}{name=knipsX,description={Ist der Name unseres Programms - er leitet sich von "`Fotos knipsen"' und "`Exif"' ab}}

\newglossaryentry{exif}{name=Exif,description={Exchangeable Image File Format ist ein Standard der Japan Electronic and Information Technology Industries Association (JEITA) für das Dateiformat, in dem moderne Digitalkameras Informationen über die aufgenommenen Bilder (Metadaten) speichern}}

\newglossaryentry{Lichtformer}{name=Lichtformer,description={Lichtformer sind Hilfsmittel für Blitz- oder Dauerlicht-Anlagen in der Fototechnik zu Steuerung der Lichtcharakteristik}}

\newglossaryentry{jpg}{name=JPEG,description={Das JPEG File Interchange Format (JFIF) ist ein 1991 von Eric Hamilton entwickeltes Grafikformat zur Speicherung von Bildern, die nach der JPEG-Norm komprimiert wurden. Als Dateinamenserweiterung wird meistens .jpg, seltener .jpeg, .jpe oder .jfif verwendet}}

\newglossaryentry{lexgraph}{name=lexikographisch,description={Die lexikographische Ordnung ist in der Informatik und Mathematik eine Methode, um aus einer linearen Ordnung für einfache Objekte (beispielsweise Buchstaben angeordnet nach dem Alphabet) eine lineare Ordnung für zusammengesetzte Objekte (beispielsweise Wörter) zu erhalten. Das namengebende Beispiel ist die Anordnung der Wörter in einem Lexikon: Sie werden zunächst nach ihren Anfangsbuchstaben sortiert, dann die Wörter mit gleichen Anfangsbuchstaben nach dem jeweils zweiten Buchstaben usw. Ist ein Wort ganz in einem anderen als Anfangsteil enthalten (wie beispielsweise „Tal“ in „Talent“), so wird das kürzere Wort zuerst aufgeführt. In desem Programm, wird vom Buchstaben "`a"' ausgehend sortiert}}

\newglossaryentry{dragndrop}{name=Drag \& Drop,description={Drag \& Drop, deutsch "`Ziehen und Fallenlassen"', ist eine Methode zur Bedienung grafischer Benutzerschnittstellen von Rechnern durch das Bewegen grafischer Elemente mittels eines Zeigegerätes. Ein Element, wie z. B. eine Datei kann damit gezogen und über einem möglichen Ziel losgelassen werden. Im Allgemeinen kann Drag \& Drop genutzt werden, um Aktionen auszuführen oder Beziehungen zwischen zwei abstrakten Objekten herzustellen}}


\newglossaryentry{wilcoxon}{name=Wilcoxon-Test,description={Der Wilcoxon-Test ist ein parameterfreier statistischer Test. Er dient zur Überprüfung der Signifikanz der Übereinstimmung zweier Verteilungen, also ob zwei unabhängige Verteilungen A und B zu derselben Grundgesamtheit gehören}}


\newglossaryentry{pwert}{name=p-Wert,description={Der p-Wert ist eine Kennzahl zur Auswertung von statistischen Tests. Er steht in enger Beziehung mit dem Signifikanzniveau, lässt sich aber nicht so einfach in Tabellen fassen, sodass die praktische Anwendung erst mit Einführung von Computern und Statistik-Software möglich geworden ist}}

\newglossaryentry{dateimanager}{name=Dateimanager,description={Ein Dateimanager ist ein Computerprogramm, mit dem man den Inhalt von Datenträgern auf einem Computer ansehen und bearbeiten kann. Grundfunktionen sind das Auflisten aller Dateien in einem Verzeichnis sowie die Darstellung der Verzeichnisstruktur. Man kann Dateien und Verzeichnisse suchen, verschieben, kopieren, umbenennen, löschen, ihre Attribute ändern und neue Verzeichnisse anlegen}}

\newglossaryentry{java3d}{name=Java 3D,description={Java 3D ist eine Klassenbibliothek von Java-Klassen zur Erzeugung, Manipulation und Darstellung dreidimensionaler Grafiken innerhalb von \\ Java-Anwendungsprogrammen und -Applets. Mit Java 3D können also durch ein Java-Programm dreidimensionale Objekte modelliert, gerendert sowie das Verhalten und die Ansicht gesteuert werden.}}

