% F�r die Description muss man den Satz nicht mit einem Punkt . abschlie�en, sonst werden unter Umst�nden zwei Punkte .. im Glossar angezeigt.

\newglossaryentry{kit}{name=KIT,description={Karlsruher Institut f�r Technologie ist die neue Bezeichnung der fr�heren Universit�t Karlsruhe (TH), die seit dem 01.10.2009 ihre G�ltigkeit hat},first={Karlsruher Institut f�r Technologie (vormals Universit�t Karlsruhe (TH))}}

\newglossaryentry{ipd}{name=IPD Tichy,description={Institut f�r Programmstrukturen und Datenorganisation am Karlsruher Institut f�r Technologie, Lehrstuhl Professor Tichy},first={Institut f�r Programmstrukturen und Datenorganisation am Karlsruher Institut f�r Technologie, Lehrstuhl Professor Tichy}}

\newglossaryentry{ps}{name=picture set,description={Ist eine Bildmenge, die Bilder, Verzeichnisse und/oder weitere Bildmengen enthalten kann},first={picture set (dt: Bildmenge)}}

\newglossaryentry{rp}{name=Report,description={Ist eine Auswertung, die Bildmengen mit Diagrammtypen verkn�pft},first={Report (dt: Auswertung)}}

\newglossaryentry{tempX}{name=Programm,description={Programm},first={Programm}}

\newglossaryentry{exif}{name=Exif,description={Exchangeable Image File Format ist ein Standard der Japan Electronic and Information Technology Industries Association (JEITA) f�r das Dateiformat, in dem moderne Digitalkameras Informationen �ber die aufgenommenen Bilder (Metadaten) speichern}}

