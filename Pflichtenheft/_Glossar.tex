% Für die Description muss man den Satz nicht mit einem Punkt . abschließen, sonst werden unter Umständen zwei Punkte .. im Glossar angezeigt.

\newglossaryentry{kit}{name=KIT,description={Karlsruher Institut für Technologie ist die neue Bezeichnung der früheren Universität Karlsruhe (TH), die seit dem 01.10.2009 ihre Gültigkeit hat},first={Karlsruher Institut für Technologie (vormals Universität Karlsruhe (TH))}}

\newglossaryentry{ipd}{name=IPD Tichy,description={Institut für Programmstrukturen und Datenorganisation am Karlsruher Institut für Technologie, Lehrstuhl Professor Tichy},first={Institut für Programmstrukturen und Datenorganisation am Karlsruher Institut für Technologie, Lehrstuhl Professor Tichy}}

\newglossaryentry{ps}{name=picture set,description={Ist eine Bildmenge, die Bilder, Verzeichnisse und/oder weitere Bildmengen enthalten kann},first={picture set (dt: Bildmenge)}}

\newglossaryentry{rp}{name=Report,description={Ist eine Auswertung, die Bildmengen mit Diagrammtypen verknüpft},first={Report (dt: Auswertung)}}

\newglossaryentry{tempX}{name=Programm,description={Programm},first={Programm}}

\newglossaryentry{exif}{name=Exif,description={Exchangeable Image File Format ist ein Standard der Japan Electronic and Information Technology Industries Association (JEITA) für das Dateiformat, in dem moderne Digitalkameras Informationen über die aufgenommenen Bilder (Metadaten) speichern},first={Exchangeable Image File Format}}

