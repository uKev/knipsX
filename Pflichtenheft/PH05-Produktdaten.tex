\section{Produktdaten}

\subsection{Programmdaten}

\label{subsec:programmdaten}

\begin{description}
	
	\item[/D010/] \textit{Daten die im Programm gespeichert sind:}
	
	\begin{itemize} 
			
			\item Alle im Programm verfügbaren Projekte (in dem Projektverzeichnis - siehe Kapitel \ref{subsec:projektmanagement})
			
			\item Alle Auswertungen eines aktiven Projektes
			
			\item Alle Bildmengen eines aktiven Projektes (damit verbunden alle Pfade zu den Bildern, die in den Bildmengen liegen)
			
			\item \gls{exif}-Parameter zu jedem Bild der Bildmengen eines aktiven Projekts (siehe Kapitel \ref{subsec:musskriterien})
	
	\end{itemize}

	\item[/D020/] \textit{Daten die mit einem Projekt gespeichert werden:}
	
	\begin{itemize}
		
		\item Projektid, Projektname, Projektbeschreibung, Erstellungsdatum, letztes Bearbeitungsdatum
		
		\item Alle zu einem Projekt gehörende Bildmengen, Verzeichnispfade und/oder Bildpfade
		
		\item Alle zu einem Projekt gehörenden Auswertungen
	
	\end{itemize}

	\item[/D030/] \textit{Daten die mit einer Bildmenge gespeichert werden:}
	
	\begin{itemize}
	
		\item Bildmengenid, Bildmengenname
		
		\item Vollständiger Pfad der zur Bildermenge gehörenden Verzeichnisse
		
		\item Vollständiger Pfad der zur Bildermenge gehörenden Bilder
		
		\item Vollständiger Pfad der Bilder, die ausgeschlossen werden sollen
	
	\end{itemize}
	
	\item[/D040/] \textit{Mit einer Auswertung gespeicherte Daten:}
	
	\begin{itemize}
		
		\item Auswertungsid, Auswertungsname, verknüpfte Bildmengen, \gls{exif}-Keywords, ausgewählter Diagrammtyp
		
		\item Diagrammspezifische Daten \itshape{(siehe \ref{subsec:daten-diagrammtypen})}
	
	\end{itemize}
		
\end{description}

\subsection{Daten der einzelnen Diagrammtypen}

\label{subsec:daten-diagrammtypen}

\begin{description}

	\item[/D110/] \textit{Daten des Diagrammtyps "`Boxplot"':}

\end{description}