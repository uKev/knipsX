\section{Produktdaten}

\subsection{Programmdaten}

\begin{itemize}
	\item /D10/\\Daten die im Programm gespeichert sind /LD10/:
	\begin{itemize} 
		\item Alle im Programm verfügbaren Projekte, deren Name sowie deren Speicherort
		\item Alle Projekte müssen mit einem Bearbeitungsdatum versehen werden
	\end{itemize}

	\item /D20/\\Daten die mit einem Projekt gespeichert werden /LD20/:
	\begin{itemize}
		\item Projektname, letztes Bearbeitungsdatum
		\item Alle zu einem Projekt gehörende Bildmengen, Verzeichnissepfade und/oder Bildpfade
		\item Alle zum Projekt gehörenden Auswertungen
	\end{itemize}

	\item /D30/\\Daten die mit einer Bildmenge gespeichert werden /LD30/:
	\begin{itemize}
		\item Bildmengenname
		\item Hauptverzeichnis der zur Bildermenge gehörenden Verzeichnisse
		\item Vollständiger Pfad der zur Bildermenge gehörenden Bilder
		\item Vollständiger Pfad der Bilder, die ausgeschlossen werden sollen
	\end{itemize}
	
	\item /D40/\\Mit einer Auswertung gespeicherte Daten /LD40/:
	\begin{itemize}
		\item Auswertungsname, Bildmengen, Ausgewählter Diagrammtyp, Auswertungseinstellung, Bildmengentag
		\item Diagrammspezifische Daten \itshape{(siehe \ref{subsec:daten-diagrammtypen})}
	\end{itemize}
		
	\item /D50/\\ Mit der Statistik gespeicherte Daten /LD50/:
	\begin{itemize}
		\item Erstellen eines JPEGs aus der aktuellen Statistik.
	\end{itemize}
\end{itemize}

\subsection{Daten der einzelnen Diagrammtypen}
\label{subsec:daten-diagrammtypen}

\begin{itemize}
	\item /D60/ Daten des Diagrammtyps "`Boxplot"' /LD60/:
\end{itemize}