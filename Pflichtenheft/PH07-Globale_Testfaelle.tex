\section{Globale Testfälle}

Die Testfälle sollen sowohl auf Windows als auch Linux ausgeführt und überprüft werden.

 % TODO: Ein Testfall für jede Anforderung. *bei Fertigstellung diese Zeile entfernen*
 % TODO: Testfallnummern am Ende überprüfen und ggf. den Anforderungsnummern anpassen.

\subsection{Testfälle für funktionale Anforderungen}
	
	\subsubsection{Programmausführung}
	
		\begin{description}

			\item[/T010/] \textit{Programm beenden:}\par Das Programm wird während in der Projektübersicht beendet.
			\item[/T011/] Das Programm wird in der Projektansicht beendet.
				
			\item[/T020/] \textit{Projekt speichern:}\par Das in /T110/ erstellte Projekt wird mithilfe des entsprechenden Knopfes innerhalb der Projektansicht gespeichert. 
				
			\item[/F030/] \textit{Automatische Anpassung der Größe der Bedienoberfläche:}\par TODO: Wie soll man das testen?	
				
			\item[/F040/] \textit{Automatisches durchsuchen des Projektordners:}\par Das Programm wird gestartet und es wird überprüft ob die generierte Projektliste mit den bisher erstellten Projekten übereinstimmt.
		
		\end{description}

	\subsubsection{Projektmanagement}
	
		\begin{description}
		
			\item[/T110/] \textit{Neues Projekt mit Namen erstellen:}\par Es wird in der Projektübersicht ein neues Projekt mit dem Namen "`Schwarzwald \#3 mit Kamera \$B54\% ~n @ 2. \& 3. Berg"' erstellt.
			\item[/T111/] \textit{Namenskollision: }Es wird zusätzlich ein Projekt mit dem Namen "David, Andreas - ein Vergleich" erstellt und gespeichert. Im Anschluss wird noch einmal versucht ein Projekt mit dem gleichen Namen erstellt. Dies schlägt fehl.
				
			\item[/T120/] \textit{Projekt aktivieren:}\par TODO: was soll hier getestet werden?
				
			\item[/T130/] \textit{Gespeichertes Projekt öffnen:}\par Das in \textbf{/T110/} gespeicherte Projekt wird in der Projektübersicht ausgewählt und geöffnet. Dabei werden die Exif-Daten der Bilder in den Bildmengen des Projekts eingelesen und im Anschluss überprüft.

			\item[/T131/] \textit{Ansprechbarkeit:}	Während dem Einlesen muss das Programm auf weitere Eingaben reagieren.
			
			\item[/T132/] \textit{Vollständigkeit:} Nach dem Einlesen muss die Bilderliste, die Bildmengenliste, die Verzeichnisliste und die Auswertungsliste vollständig sein und alle diesem Projekt zugeordneten Objekte beinhalten.
			
			\item[/T140/] \textit{Projekt kopieren:}\par Es wird das Projekt aus \textbf{/T110/} kopiert. Dabei wird der Kopie der Namen "David, Andreas - Vergleich++" gegeben. Es wird kontrolliert ob alle Daten des Quellprojekts - mit Ausnahme des exakten Namens - auch in der Kopie vorhanden sind.
				
			\item[/T150/] \textit{Projekt entfernen:}\par Das in \textbf{/T140/} kopierte Projekt wird gelöscht. Es wird überprüft ob alle zugehörigen Daten im Projektordner ebenfalls gelöscht sind.
		
		\end{description}
	
	\subsubsection{Bildmengenmanagement}
		
		\begin{description}
		
			\item[/T210/] \textit{Neue Bildmenge erstellen:}\par Es wird im Projekt aus \textbf{/T110/} eine Bildmenge mit dem Namen "Berg 2" erstellt. Dieser Bildmenge wird das Verzeichnis "Skihütte Berg 2" zugeordnet.
				
			\item[/T220/] \textit{Bildmenge auswählen:}\par Es wird die in \textbf{/T210/} erstellte Bildmenge ausgewählt und kontrolliert, ob sich alle Bilder des Verzeichnisses "Skihütte Berg 2" in der Bildmenge befinden.

			\item[/T230/] \textit{Bilder per Drag \& Drop hinzufügen:}\par Es wird eine Bildmenge "Berg 3" erstellt und ihr werden per Drag \& Drop sowohl das Verzeichnis "Skilift 1 Berg 3" sowie zwei weitere Einzelbilder hinzugefügt.
			
			\item[/T231/] \textit{Bilder hinzufügen:} Es werden über die Hinzufügen-Funktion weitere 3 Einzelbilder sowie der Ordner "Abfahrt Berg 3" zur Bildmenge "Berg 3" aus \textbf{/T230/} hinzugefügt.
			
			\item[/T240/] \textit{Hinzufügen von Bildmengen zu einer vorhandenen Bildmenge:}\par Es wird eine Bildmenge "Berg 2+3" erstellt. Ihr wird die Bildmenge "Berg 3" nach Ergänzung durch \textbf{/T231/} und die Bildmenge "Berg 2" aus \textbf{/T210/} hinzugefügt.
				
			\item[/T250/] \textit{Bildmenge entfernen:}\par Es wird die Bildmenge "Berg 3" von \textbf{/T231/} entfernt. Die Bildmenge muss nun ebenfalls aus der Bildmenge "Berg 2+3" entfernt worden sein.
			
			\item[/T260/] \textit{Unterbildmengen in der Inhaltsliste:}\par Es werden 2 weitere Bildmengen zur Bildmenge "Berg 2+3" aus \textbf{/T240/} hinzugefügt. Diese müssen \gls{lexgraph} sortiert in der Inhaltsliste sein.
			\item[/T261/] \textit{Ordner in der Inhaltsliste:}\par Zur Bildmenge aus \textbf{/T260/} werden 3 weitere Ordner mit Bildern hinzugefügt. Diese müssen in sich \gls{lexgraph} sortiert sein und unterhalb der Unterbildmengen stehen.
			\item[/T262/] \textit{Bilder in der Inhaltsliste:}\par Zur Bildmenge aus \textbf{/T261/} werden 10 weitere Bilder hinzugefügt. Diese müssen in sich \gls{lexgraph} sortiert sein und sich alle unterhalb der Ordner in der Inhaltsliste befinden.
			\item[/T263/] \textit{Farbe der Inhalte:}\par Basierend auf \textbf{/T262/. Die 3 Blöcke aus \textbf{/T260/}, \textbf{/T261/} und \textbf{/T262/} müssen unterschiedliche Textfarben haben. 
		
		\end{description}
	
	\subsubsection{Diagrammmanagement}

		TODO: Wie soll man Diagramme testen?	
	
		\begin{description}
			
			\item[/T310/] \textit{Tabelle:}\par 

			\item[/T320/] \textit{Histogram 2D:}\par 
		
			\item[/T330/] \textit{Histogram 3D:}\par 

			\item[/T340/] \textit{Boxplot:}\par 

			\item[/T350/] \textit{Punktewolke:}\par 
			

			
		\end{description}
	
	\subsubsection{Auswertungsmanagement}

		\begin{description}
		
			\item[/T410/] \textit{Neue Auswertung anlegen.}\par Es wird eine neue Auswertung ohne zugeordnete Bildmenge mit Namen "Vergleich Berg 2 und Berg 3" erstellt. Als Diagrammtyp wird "Boxsplot ausgewählt"
			
			\item[/T420/] \textit{Eine bereits angelegte Auswertung über die Auswertungs-Liste auswählen.}\par
			
			\item[/T430/] \textit{Bearbeiten einer Auswertung:}\par 
				
			\item[/T440/] \textit{Eine Auswertung wird gelöscht.}\par
			
			TODO: fällt das nochfolgende komplett Weg? Gibt es Ersatz dafür?
			\item[/T110/] \textit{Eine Bildmenge wird einem bereits erstellten \gls{rp} hinzugefügt und wieder entfernt}\par
				
			\item[/T120/] \textit{Bildmengen werden beim Hinzufügen zum \gls{rp} über Dateinamen und \gls{exif} Daten gefiltert}\par
				
			\item[/T130/] \textit{Erstellung eines \gls{rp}s für jeden \gls{rp}-Typ.}\par
			
			\item[/T140/] \textit{Anzeige der Diagrammvorschau bei der Auswahl eines \gls{rp}s. }\par
			ENDE des nachfolgenden.
				
		\end{description}
	
	\subsubsection{Exif-Auswertung}
	
		\begin{description}

			\item[/T510/] \textit{Extraktion von \gls{exif} Daten:}\par 

		\end{description}
		
\subsection{Testfälle für nicht funktionale Anforderungen}
	
	\begin{description}
		
			\item[/TN010/]\textit{10.000 Fotos mit \gls{tempX} analysieren. Für das Einlesen und Extrahieren der \gls{exif} Daten dürfen maximal 25 Minuten benötigt}\par werden.
	
	\end{description}
 