\section{Globale Testfälle}

TODO: Ein Testfall für jede Anforderung. *bei Fertigstellung diese Zeile entfernen*
TODO: Testfallnummern am Ende korrigieren und den Anforderungsnummern anpassen.

\subsection{Testfälle für funktionale Anforderungen}
	
	\subsubsection{Programmausführung}
	
		\begin{description}
				
				\item[/T010/] \textit{Neues Projekt mit einem Namen erstellen.}\par Es wird in der Projektübersicht ein neues Projekt mit dem Namen "`Schwarzwald \#3 mit Kamera \$B54\% ~n @ 3. \& 4. Berg"' erstellt.
				
				\item[/T020/] \textit{Projekt speichern}\par Das in /T10/ erstellte Projekt wird gespeichert.
				
				\item[/T030/] \textit{Gespeichertes Projekt aus einer Liste auswählen, öffnen und löschen.}\par Das in /T20/ gespeicherte Projekt wird in der Projektübersicht ausgewählt und geöffnet.
				
				\item[/T031/]  Das in /T31/ geöffnete Projekt wird gelöscht.
				
				\item[/T040/] \textit{Neue Bildmenge mit Namen erstellen.}\par
				
				\item[/T050/] \textit{Eine Bildmenge aus einer Liste auswählen.}\par
				
				\item[/T060/] \textit{Eine gespeicherte Bildmenge löschen.}\par
				
				\item[/T070/] \textit{Bildmengen per Drag \& Drop aus Bilddateien und Ordnern erzeugen.}\par
				
				\item[/T080/] \textit{Neuer \gls{rp} mit Namen anlegen.}\par
				
				\item[/T090/] \textit{Ein bereits angelegter \gls{rp} über die \gls{rp}-Liste.}\par
				
				\item[/T100/] \textit{Ein \gls{rp} wird gelöscht.}\par
				
				\item[/T110/] \textit{Eine Bildmenge wird einem bereits erstellten \gls{rp} hinzugefügt und wieder entfernt}\par
				
				\item[/T120/] \textit{Bildmengen werden beim Hinzufügen zum \gls{rp} über Dateinamen und \gls{exif} Daten gefiltert}\par
				
				\item[/T130/] \textit{Erstellung eines \gls{rp}s für jeden \gls{rp}-Typ.}\par
				
				\item[/T140/] \textit{Anzeige der Diagrammvorschau bei der Auswahl eines \gls{rp}s. }\par
		
		\end{description}

	\subsubsection{Projektmanagement}
	
		\begin{description}
		
			\item[Blub]
		
		\end{description}
	
	\subsubsection{Bildmengenmanagement}
		
		\begin{description}
		
			\item[Blub]
		
		\end{description}
	
	\subsubsection{Diagrammmanagement}
	
		\begin{description}
			
			\item[Blub]
			
		\end{description}
	
	\subsubsection{Auswertungsmanagement}

		\begin{description}

			\item[Blub]

		\end{description}
	
	\subsubsection{Exif-Auswertung}
	
		\begin{description}

			\item[Blub]

		\end{description}
	
\subsection{Testfälle für nicht funktionale Anforderungen}
	
	\begin{description}
		
		\item[/T500/]\textit{10.000 Fotos mit \gls{tempX} analysieren. Für das Einlesen und Extrahieren der \gls{exif} Daten dürfen maximal 25 Minuten benötigt}\par werden.
	
	\end{description}

	\subsubsection{Testfälle für Produktumgebung}

		\begin{description}

			\item[Blub]

		\end{description}