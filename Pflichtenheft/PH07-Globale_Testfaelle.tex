\section{Globale Testfälle}

TODO: Ein Testfall für jede Anforderung. *bei Fertigstellung diese Zeile entfernen*
TODO: Testfallnummern am Ende korrigieren und den Anforderungsnummern anpassen.

\subsection{Testfälle für funktionale Anforderungen}
	
	\subsubsection{Programmausführung}
	
		\begin{description}

			\item[/F010/] \textit{Programm beenden:}\par TODO: beschreiben		
				
			\item[/T020/] \textit{Projekt speichern}\par Das in /T10/ erstellte Projekt wird gespeichert. TODO: automatisches Speichern?
				
			\item[/F030/] \textit{Automatische Anpassung der größe der Bedienoberfläsche:}\par TODO: beschreiben	
				
			\item[/F040/] \textit{Automatisches durchsuchen des Projektordners:}\par TODO: beschreiben	
		
		\end{description}

	\subsubsection{Projektmanagement}
	
		\begin{description}
		
			\item[/T110/] \textit{Neues Projekt mit einem Namen erstellen.}\par Es wird in der Projektübersicht ein neues Projekt mit dem Namen "`Schwarzwald \#3 mit Kamera \$B54\% ~n @ 3. \& 4. Berg"' erstellt.
				
			\item[/T120/] \textit{Projekt aktivieren:}\par 
				
			\item[/T130/] \textit{Gespeichertes Projekt öffnen}\par Das in \textbf{/T110/} gespeicherte Projekt wird in der Projektübersicht ausgewählt und geöffnet.
			\item[/T140/] \textit{Projekt kopieren:}\par 
				
			\item[/T150/] \textit{Projekt entfernen:}\par Das in \textbf{/T130/} geöffnete Projekt wird gelöscht.
		
		\end{description}
	
	\subsubsection{Bildmengenmanagement}
		
		\begin{description}
		
			\item[/T210/] \textit{Neue Bildmenge mit Namen erstellen.}\par
				
			\item[/T220/] \textit{Eine Bildmenge aus einer Liste auswählen.}\par

			\item[/T230/] \textit{Bildmengen per Drag \& Drop aus Bilddateien und Ordnern erzeugen.}\par
			
			\item[/F240/] \textit{Hinzufügen von Bildmengen zu einer vorhandenen Bildmenge:}\par
				
			\item[/T250/] \textit{Eine gespeicherte Bildmenge löschen.}\par
			
			\item[/T260/] \textit{Aufbau der Inhaltsliste:}\par 	
		
		\end{description}
	
	\subsubsection{Diagrammmanagement}
	
		\begin{description}
			
			\item[/T310/] \textit{Tabelle:}\par 

			\item[/T320/] \textit{Histogram 2D:}\par 
		
			\item[/T330/] \textit{Histogram 3D:}\par 

			\item[/T340/] \textit{Boxplot:}\par 

			\item[/T350/] \textit{Punktewolke:}\par 
			

			
		\end{description}
	
	\subsubsection{Auswertungsmanagement}

		\begin{description}
		
			\item[/T410/] \textit{Neue Auswertung mit Namen anlegen.}\par
			
			\item[/T420/] \textit{Eine bereits angelegte Auswertung über die Auswertungs-Liste auswählen.}\par
			
			\item[/T430/] \textit{Bearbeiten einer Auswertung:}\par 
				
			\item[/T440/] \textit{Eine Auswertung wird gelöscht.}\par
			
			TODO: fällt das nochfolgende komplett Weg? Gibt es Ersatz dafür?
			\item[/T110/] \textit{Eine Bildmenge wird einem bereits erstellten \gls{rp} hinzugefügt und wieder entfernt}\par
				
			\item[/T120/] \textit{Bildmengen werden beim Hinzufügen zum \gls{rp} über Dateinamen und \gls{exif} Daten gefiltert}\par
				
			\item[/T130/] \textit{Erstellung eines \gls{rp}s für jeden \gls{rp}-Typ.}\par
			
			\item[/T140/] \textit{Anzeige der Diagrammvorschau bei der Auswahl eines \gls{rp}s. }\par
			ENDE des nachfolgenden.
				
		\end{description}
	
	\subsubsection{Exif-Auswertung}
	
		\begin{description}

			\item[/T510/] \textit{Extraktion von \gls{exif} Daten:}\par 

		\end{description}
	
\subsection{Testfälle für nicht funktionale Anforderungen}
	
	\begin{description}
		
			\item[/T500/]\textit{10.000 Fotos mit \gls{tempX} analysieren. Für das Einlesen und Extrahieren der \gls{exif} Daten dürfen maximal 25 Minuten benötigt}\par werden.
	
	\end{description}

	\subsubsection{Testfälle für Produktumgebung}

		\begin{description}

			\item[/T510/] \textit{Extraktion von \gls{exif} Daten:}\par 

		\end{description}