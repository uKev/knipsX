\section{Funktionale Anforderungen}
\begin{itemize}
\item /F10/Projekt erstellen: benutzer muss eine Moeglichkeit haben einen Projekt zu erstellen und damit zu arbeiten. Verfuegbare funktionen:\\
	\begin{itemize}
		\item Verzeichnis(e) in Liste hinzufuegen\\
		\item Datei(en) in Liste hinzufuegen\\
		\item Datei(en) von Liste loeschen\\
		\item Verzeichnisse/Dateien zwischen Listen verschieben\\
		\item Projekt speichern\\
		\item Projekteingenschaften aendern\\
	\end{itemize}

\item /F20/EXIF Evaluierung erstellen: Dateien die in der Liste sind werden nach EXIF Daten gescannt und analysiert. Man arbeitet mit Datensaetze, die von Mengen von EXIF headers erstellt wird. \\
	\begin{itemize}
		\item Datensatz hinzufuegen\\
		\item Datensatz entfernen\\
		\item Datensatz begrenzen\\
	\end{itemize}

\item /F30/Graph (PLOT) erstellen: aus vorhandenen Datensaetzen wird einen Graph erstellt. Benutzer hat folgende Moeglichkeiten:
	\begin{itemize}
		\item Graphtyp auswaehlen\\
		\item Datensatz entfernen\\
		\item Fehlerhafte Punkte im Datensatz korrigieren/entfernen\\
		\item Achsen definieren\\
		\item Graph loeschen\\
	\end{itemize}
	
\item /F40/Graph (PLOT) aendern: Benutzer hat die Moeglichkeit bereits erstellte Graphen zu aendern. Dabei sind folgende Moeglichkeiten vorhanden:
	\begin{itemize}
		\item Graphtyp aendern\\
		\item Datensatz hizufuegen/entfernen\\
		\item Skala aendern, Messeinheit aendern\\
	\end{itemize}

\item /F50/Projekt speichern\\
	\begin{itemize}
			
		\item Graph speichern\\
			\begin{itemize}
					\item JPG Format\\
			\end{itemize}		
	\end{itemize}
\end{itemize}