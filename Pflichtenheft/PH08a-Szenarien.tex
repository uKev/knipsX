\subsection{Szenarien}

\begin{enumerate}
	\item Bernhardt arbeitet in einem Fotostudio und möchte für ein Fotoshooting am nächsten Mittwoch eine statistische Auswertung erstellen. Dabei will er feststellen, ob sich die automatische Verschlusszeit seiner Kamera mit verschiedenen \gls{Lichtformer}n ändert.\par
	Er öffnet \gls{tempX} und legt ein neues Projekt an. Nun gibt er seinem Projekt einen aussagekräftigen Namen und erstellt darauhin eine neue Auswertung, indem er auf die entsprechende Schaltfläche klickt. In dem sich öffnenden Fenster, wählt er den Diagrammtypen 2D Histogramm aus und klickt auf "`Vorwärts"', um den Einrichtungsassistenten zu starten. Als x-Achse wählt er "`shutter speed"' (dt.: Verschlusszeit) aus dem Aufklappmenü aus und klickt auf "`Vorwärts"' bis sich der Einrichtungsassistent beendet. Schließlich wird ihm angezeigt, dass er keine Bildmenge mit der aktuellen Auswertung verknüpft hat. Er klickt auf "`Speichern"' und beendet \gls{tempX}. 

	\item Thomas öffnet \gls{tempX} und erstellt ein neues Projekt mit dem Namen "`blendevsiso"'. In eine neue Bildmenge namens "`Urlaub gesamt"', fügt er die Verzeichnisse "`Urlaub 2008"' und "`Urlaub 2009"' ein. Zusätzlich fügt er der Bildmenge zwei einzelne Bilder des Verzeichnisses "`Urlaub 1998"' hinzu.	Nun erstellt er eine neue Auswertung, indem er auf die entsprechende Schaltfläche klickt.\par
	In dem sich öffnenden Fenster, vergibt er den Namen "`Blende vs Iso"' und wählt als Diagrammtyp Punktewolke. Anschließend klickt er auf "`Vorwärts"', um den Einrichtungsassistenten zu starten. Dabei wird der x-Achse der \gls{exif}-Parameter "`Blende"' und der z-Achse \gls{exif}-Parameter "`ISO"' zugewiesen . Thomas schaut sich die Punktewolke an und rotiert das Diagramm so, dass für ihn notwendige Informationen gut sichbar sind. Er wählt einige für ihn interessante Punkte auf der Diagrammfläche aus und bekommt Informationen über dieses Bild angezeigt.\par
	Schließlich vergrößert er den für ihn benötigten Bereich im Diagramm und speichert die aktuelle Ansicht als Bild im \gls{jpg}-Format. Danach, schließt er das Auswertungsfenster, speichert das Projekt und beendet das Programm.

	\item Vor einer Woche hat Thomas ein Kurzurlaub in Spanien gemacht. Das Verzeichnis "`Urlaub 2009"' besitzt somit neue Bilder. Nun will Thomas seine bereits definierte Auswertung mit den neu vorhandenen Bilder erneut durchführen.\par
	Er startet \gls{tempX} und öffnet das vorhandenen Projekt mit dem Namen "`blendevsiso"'. Da sich die Bilder in der Bildmenge geändert haben, muss die Bildmenge aktualisiert werden. Dafür drückt Thomas den Knopf "`Aktualisieren"' im linken unteren Teil der Projektansicht.  Somit ist die Bildmenge aktualisiert. Nun wählt er aus der Auswertungsliste seine bereits definierte Auswertung aus und lässt sie sich neu anzeigen. Zufrieden schließt er den Auswertungsdialog, speichert das Projekt und beendet das Programm.

	\item Rebecca hat von ihrem Freund Thomas ein tolles Auswertungsprogramm geschenkt bekommen, mit dem ihr größter Wunsch ihn Erfüllung geht: Sie wollte schon immer einmal wissen, an welchem Tag sie am meisten Fotos mit ihrer Kamera gemacht hat.\par
	Flugs hat sie das Programm eingerichtet. Nach dem Programmstart, erstellt sie ein neues Projekt "`Fotos pro Tag"' und öffnet die Projektansicht. Aufgrund ihrer ausgeprägten analytischen Fähigkeiten, findet sie sofort die Funktion zum Erstellen einer Auswertung. Sie vergibt der Auswertung den Namen "`Auswertung 1"' und schaut sich erst einmal in Ruhe die verschiedenen Diagrammtypen an.\par
	Nach längerer Bedenkzeit, hat sie sich für das 2D Histogramm entschieden. Per Assistent, hat sie in Windeseile alle notwendigen Einstellungen getroffen, allerdings fällt ihr auf, dass sie noch gar keine Bildmenge definiert hat. Sie übernimmt die aktuellen Einstellungen der Auswertung und erstellt eine Bildmenge mit Namen "`alle"'. In diese Bildmenge, fügt sie alle ihre Bilder hinzu, in dem sie das Elternverzeichnis auswählt. Nun bearbeitet sie die gespeicherte Auswertung, und fügt ihr die Bildmenge hinzu. Geschafft - endlich kann sie sich die gewünschte Information anschauen. Freudenstrahlend schließt sie den Auswertungsdialog, speichert das Projekt und beendet das Programm.
	
	\item Markus Müller benutzt \gls{tempX} schon seit einigen Monaten. Nun will er einige nicht mehr benötigte Projekte entfernen. Er startet das Programm und entfernt einzeln Projekte, die ihn nicht mehr interessieren. Nach dem Entfernen stellt er fest, dass er versehentlich ein wichtiges Projekt für den morgigen Tag gelöscht hat. Glücklicherweise, hat er noch ein ähnliches Projekt vom vorigen Monat. Er aktiviert das alte Projekt und kopiert es. Nun ändert er einige Kleinigkeiten, speichert das Projekt und kann wieder beruhigt schlafen gehen.
	
	\item Heiner will sich eine neue Kamera kaufen und ist sich noch nicht sicher, ob er eine Kamera mit hohen oder niedrigen ISO-Werten kaufen soll. Er setzt sich an den Rechner Schwester Rebecca und versucht mit dem Programm \gls{tempX} eine neue Auswertung zu erstellen. Vorher schließt er noch seine externe Festplatte an den Rechner an, um seine Daten verfügbar zu haben.\par
	Da Heiner ein IT-Profi ist, hat er sich schnell zum Auswertungsdialog durchgekämpft. Er erstellt einen Boxplott, mit dem er seine 9.345 Bilder nach dem ISO-Wert statistisch aufbereitet sehen will. Hochzufrieden weiß er nun, dass er eine Kamera kaufen sollte, die hohe ISO-Werte unterstützt. Er schließt nun noch den Auswertungsdialog und beendet das Programm, ohne sein Projekt zu speichern.
 
\end{enumerate}