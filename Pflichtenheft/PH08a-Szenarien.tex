\subsection{Szenarien}

\begin{itemize}
	\item Bernhardt arbeitet in einem Fotostudio und möchte für ein Fotoshooting am nächsten Mittwoch eine statistische Auswertung erstellen. Dabei will er feststellen, ob sich die automatische Verschlusszeit seiner Kamera mit verschiedenen \gls{Lichtformer} ändert.
	
Er öffnet \gls{tempX} und legt ein neues Projekt an. Er gibt seinem Projekt einen aussagekräftigen Namen.  Daraufhin erstellt er eine neue Auswertung, indem er auf die entsprechende Schaltfläche klickt. In dem sich öffnenden Fenster wählt er den Diagrammtypen 2D-Histogramm aus und klickt auf ''Weiter'', um den Einrichtungsassistenten zu starten. Als x-Achse wählt er ''shutter speed'' (dt.: Verschlusszeit) aus dem Aufklappmenü aus und klickt auf ''Weiter'' bis sich der Einrichtungsassistent beendet. Schließlich wird ihm angezeigt, dass er keine Bildmenge mit der aktuellen Auswertung verknüpft hat. Er klickt auf ''Speichern'' und beendet \gls{tempX}. 

	\item Thomas öffnet \gls{tempX}, erstellt ein neues Projekt mit Namen "`belichtungvsiso"'. In die neue Bildmenge fügt er die Verzeichnisse "`Urlaub 2008"' und "`Urlaub 2009"' ein. Zusätzlich nimmt er zwei einzelne Bilder vom Verzeichnis "`Urlaub 1998"'. Thomas erstellt eine neue Auswertung von eine vorhandene Bildmenge. Er waehlt 3-D Punktewolke und startet eine Diagrammassistent. Dabei X-Achse wird zu Datum zugewiesen. Y- und Z-Achse werden zu Belichtungszeit und ISO-Wert entsprechend zugewiesen. Thomas schaut 3-D Diagramm an. Dabei rotiert er die Diagramm so, dass noetige Informationen gut sichbar sind. Er waehlt die Punkte auf Diagramm die für ihm interessant sind und sieht ein entsprechende Bild in Kleinformat. Schließlich vergrößert er benötigte Bereich in Diagramm und speichert sie als JPG Datei. Thomas beendet \gls{tempX}. 

\item In eine Woche hat Thomas ein Kurzurlaub in Spanien gemacht. So ist die Verzeichnis "`Urlaub 2009"' mit neue Bilder eingereicht. Nun will Thomas dasselbe Auswertung zusammen mit neu vorhandenen Bilder machen. Thomas startet das Programm \gls{tempX} und öffnet ein vorhandenen Projekt mit dem Namen "`belichtungvsiso"'. Da die Bilder im Bildmenge geändert sind, muss die Bildmenge aktualisiert werden. Dafuer druckt Thomas den Knopf "`Aktualisieren"' im linke untere Teil von Projektansichtsfenster.  Somit ist die Bildmenge aktualisiert. Thomas wählt Diagramm "`3-D Punktewolke"'. Die Einstellungen in Diagrammassistent waren gespeichert. Deswegen erstellt Thomas die Diagramm sofort. Er rotiert die Diagramm um richtigen Winkel zu finden. Als er ein noetige Winkel ausgewählt hat, speichert er die Diagramm in JPG Datei und vergleicht sie mit vorigem Ergebnis. 
\end{itemize}