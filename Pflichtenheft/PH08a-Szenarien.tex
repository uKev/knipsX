\subsection{Szenarien}

\begin{itemize}
	\item Bernhardt arbeitet in einem Fotostudio und möchte für ein Fotoshooting am nächsten Mittwoch eine statistische Auswertung erstellen. Dabei will er feststellen, ob sich die automatische Verschlusszeit seiner Kamera mit verschiedenen \gls{Lichtformer} ändert.
	
Er öffnet \gls{tempX} und legt ein neues Projekt an. Er gibt seinem Projekt einen aussagekräftigen Namen.  Daraufhin erstellt er eine neue Auswertung, indem er auf die entsprechende Schaltfläche klickt. In dem sich öffnenden Fenster wählt er den Diagrammtypen 2D-Histogramm aus und klickt auf ''Weiter'', um den Einrichtungsassistenten zu starten. Als x-Achse wählt er ''shutter speed'' (dt.: Verschlusszeit) aus dem Aufklappmenü aus und klickt auf ''Weiter'' bis sich der Einrichtungsassistent beendet. Schließlich wird ihm angezeigt, dass er keine Bildmenge mit der aktuellen Auswertung verknüpft hat. Er klickt auf ''Speichern'' und beendet \gls{tempX}. 

\end{itemize}