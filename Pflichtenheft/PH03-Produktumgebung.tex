\section{Produktumgebung}

\label{sec:produktumgebung}

\gls{tempX} soll auf einem der Poolrechner im Raum 356 des Informatikbaus (Geb 50.34) des KITs laufen.

\subsection{Software}

\label{subsec:software}

	\begin{itemize}
		
		\item Betriebssystem: 

			\begin{itemize}

				\item Windows XP/Vista/7

				\item Linux (mit Fenstermanager KDE oder Gnome)

				\item (optional) Mac OS X 10.6

			\end{itemize}
	
		\item Laufzeitumgebung:
		
			\begin{itemize}
				
				\item Java 1.6
				
				\item \gls{java3d}
				
			\end{itemize}
		
	\end{itemize}
	
\subsection{Hardware}

	\begin{itemize}
		
		\item Mindestanforderung an den Arbeitsplatzrechner: 
		
			\begin{itemize}
			
				\item Dual Core 2 Ghz
				
				\item 2 GB RAM
			
				\item Bildschirm mit einer Auflösung von 720x500 Pixel
			
				\item 20 MB freier Speicherplatz auf der Festplatte
			
			\end{itemize}
	
		\item Empfohlene Anforderungen an den Arbeitsplatzrechner:
	
			\begin{itemize}
			
				\item Intel\textregistered Core\texttrademark 2 Quad Q6600 2,4 Ghz
				
				\item 8 GB RAM
				
				\item Bildschirm mit einer Auflösung größer als 720x500 Pixel
				
				\item Mehr als 20 MB freier Speicherplatz auf der Festplatte
			
			\end{itemize}	
	
		\item Kamera:

			\begin{itemize}

				\item Alle Kameramodelle, die mindestens den JEITA \gls{exif} Version 2.1 Standard vom 1. Juni 1998 einhalten

			\end{itemize}
		
	\end{itemize}