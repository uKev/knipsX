\section{Zielbestimmungen}
\begin{itemize}
  \item Fotografen sollen durch das Produkt in der Lage sein, aus Metadaten ihrer Bilder, welche dem \gls{exif}-Standard entsprechen, Statistiken über ihre Einstellungen beim Fotographieren zu erstellen, diese zu präsentieren sowie sie zu analysieren.
  \end{itemize} 
\subsection{Musskriterien}

\label{subsec:musskriterien}

\begin{itemize}
	\item Verwalten von Projekten
	\item Verwalten von Bildmengen
	\item Hinzufügen und Entfernen von Bildern zu bzw. aus Bildmengen per Ordnermenü und Drag und Drop
	\item Bildvorschau bei der Bildauswahl
	\item Beibehalten von ausgewählten Bildern nach Programmbeendigung
	\item Bild/\gls{exif} Filter zur Auswahl bestimmter Bildmengen
	\item Auslesen, Anzeigen und Auswerten von \gls{exif} Daten
	\item Auszuwertende \gls{exif} Daten:
	\label{subsec:auszuwertendedaten}
			\begin{itemize}
			\item Kameramodel
			\item Blende 
			\item Verschlusszeit
			\item ISO-Wert
			\item Brennweite
			\item Datum
			\item Wochentag
			\item Uhrzeit
			\item Objektivname
		\end{itemize}
	\item Verwalten von Auswertungen
	\item Vergleich mehrere Bildmengen
	\item Erstellen von verschieden Diagrammtypen aus \gls{exif} Daten
	\item Die verschiedenen Diagrammtypen:
		\begin{itemize}
		  \item Tabelle
			\item 2D Histogramm
			\item 3D Histogramm
			\item Punktewolke
			\item Boxplot
		\end{itemize}
	\item Exportieren bzw. Speichern von Diagrammen im JPEG Dateiformat
	\item Produkt muss in Java 1.6 geschrieben sein	
\end{itemize}

\subsection{Wunschkriterien} 
\subsubsection{Hohe Priorität}
	\begin{itemize}
	\item Internationalisierungsmechanismen vorbereiten
	\item weiter Ausgabeformate unterstützen 
	\end{itemize}
\subsubsection{Mittlere Priorität}
	\begin{itemize}
	\item Optimierung von Algorithmen
	\item Anzeige von Thumbnails sowie Dateinamen in Diagrammen über eine Mengenauswahl
	\item Vernünftige eventuell anpassbare Diagrammskalierungen
	\item Konfigurierbarkeit des Layouts	
	\end{itemize}
\subsubsection{Niedrige Priorität}
	\begin{itemize}
	\item Unterstützen weiterer \gls{exif}-Parameter sowie Kameraspezifischer Parameter
	\item Normierung von Werten, z.B. Brennweitenkorrektur
 	\item Unterstützung weiterer Bildformate mit Metadaten 	
 	\item Weitere Diagrammtypen
	\end{itemize}


\subsection{Abgrenzungskriterien} 
\begin{itemize}
	\item \gls{tempX} soll keine \gls{exif} Daten bearbeiten können.
	\item \gls{tempX} soll keine Bilder bearbeiten bzw. löschen können.
	\item \gls{tempX} soll keine Bilder ausdrucken können.
	\item \gls{tempX} soll keine Diashow anzeigen können.
	\item \gls{tempX} muss keinen hohen Sicherheitsansprüchen genügen.
\end{itemize}
