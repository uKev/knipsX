\section{Zielbestimmungen}
\begin{itemize}
  \item Fotographen sollen durch das Produkt in der Lage sein, aus Metadaten ihrer Bilder, welche dem \gls{exif}-Standard entsprechen, Statistiken �ber ihre Einstellungen beim Fotographieren zu erstellen, diese zu pr�sentieren sowie sie zu analysieren.
  \end{itemize} 
\subsection{Musskriterien} 
\begin{itemize}
	\item Verwalten von Projekten
	\item Verwalten von Bildmengen
	\item Hinzuf�gen (Entfernen) von Bildern zu (aus) Bildmengen per Ordnermen� und Drag und Drop
	\item Bildvorschau bei der Bildauswahl
	\item Beibehalten von ausgew�hlten Bildern nach Programmbeendigung
	\item Bild/\gls{exif} Filter zur Auswahl bestimmter Bildmengen
	\item Auslesen, Anzeigen und Auswerten von \gls{exif} Daten
	\item Auszuwertende \gls{exif} Daten:
			\begin{itemize}
			\item Kameramodel
			\item Blende 
			\item Verschlusszeit
			\item ISO-Wert
			\item Brennweite
			\item Datum
			\item Wochentag
			\item Uhrzeit
			\item Objektivname
		\end{itemize}
	\item Verwalten von Auswertungen
	\item Vergleich mehrere Bildmengen
	\item Erstellen von verschieden Diagrammtypen aus \gls{exif} Daten
	\item Die verschiedenen Diagrammtypen:
		\begin{itemize}
		  \item Tabelle
			\item 2D Histogramm (Zwei-Werte-Balkendiagramm)
			\item 3D Histogramm (Drei-Werte-Balkenfelddiagramm)
			\item 3D Cluster / Wolkendiagramm (Punktediagramm)
			\item Boxplots
		\end{itemize}
	\item Exportieren bzw. Speichern von Diagrammen im JPEG Dateiformat
	\item Produkt muss in Java 1.6 geschrieben sein	
\end{itemize}

\subsection{Wunschkriterien} 
\begin{itemize}
  \item Unterst�tzung weiterer Bildformate mit Metadaten
	\item Normierung von Werten, z.B. Brennweitenkorrektur
	\item Umbennenen von Bilddateinamen
	\item Unterst�tzen weiterer \gls{exif}-Parameter sowie Kameraspezifischer Parameter
	\item Fl�ssige Programmdarstellung bei angegebenen Hardwarevoraussetzungen
	\item Anzeige von Tags in den Diagrammen
	\item Weitere Diagrammtypen
	\item Anzeige von Thumbnails sowie Dateinamen in Diagrammen �ber eine Mengenauswahl
	\item Bilder von Diagrammen aus aktueller Sichtperspektive abspeichern sowie weiter Ausgabeformate unterst�tzen 
	\item Vern�nftige eventuell anpassbare Diagrammskalierungen
	\item Leichte Konfigurierbarkeit des Layouts
	\item Gute User-Usability
	\item Internationalisierung
\end{itemize}

\subsection{Abgrenzungskriterien} 
\begin{itemize}
	\item \gls{tempX} soll keine \gls{exif} Daten bearbeiten k�nnen.
	\item \gls{tempX} soll keine Bilder bearbeiten bzw. l�schen.
	\item \gls{tempX} soll keine Bilder ausdrucken.
	\item \gls{tempX} soll keine Diashow anzeigen.
	\item \gls{tempX} muss keinen sehr hohen Sicherheitsanspr�chen gen�gen.
\end{itemize}
