\section{Zielbestimmungen}

\begin{itemize}
  \item Fotografen sollen durch das Produkt in der Lage sein, aus Metadaten ihrer Bilder, welche dem \gls{exif}-Standard entsprechen, Statistiken über ihre Einstellungen beim Fotografieren zu erstellen, diese zu präsentieren sowie sie zu analysieren.
\end{itemize} 

\subsection{Musskriterien}

\label{subsec:musskriterien}

\begin{itemize}
	\item Verwaltung von Projekten
	\item Verwaltung von Bildmengen in Projekten
	\item Verwaltung von Auswertungen
	\item Auslesen, Anzeigen und Auswerten von \gls{exif}-Parametern\\\\
				Auszuwertende \gls{exif}-Parameter sind:
				\label{subsec:auszuwertendedaten}
				\begin{itemize}
					\item Kameramodel
					\item Blitz
					\item Blende 
					\item Verschlusszeit
					\item ISO-Wert
					\item Brennweite
					\item Datum
					\item Wochentag
					\item Uhrzeit
					\item Objektivname
				\end{itemize}
	\item Hinzufügen von Bildern zu Bildmengen per Dateidialog und \gls{dragndrop}
	\item Entfernen von Bildern aus Bildmengen
	\item Bei der Bildauswahl, müssen Vorschaubilder angezeigt werden
	\item Beibehalten von ausgewählten Bildmengen nach Programmbeendigung
	\item Filterung von Bilder anhand von \gls{exif}-Keywords
	\item Vergleich mehrere Bildmengen in einer Auswertung
	\item Erstellen und Anzeigen von verschieden Diagrammtypen aus Bildmengen\\\\
				Notwendige Diagrammtypen:
				\begin{itemize}
					\item Tabelle
					\item 2D Histogramm
					\item 3D Histogramm
					\item Punktewolke
					\item Boxplot \& Unterstützung des Wilcoxon-Mann-Whitney-Tests
				\end{itemize}
	\item Exportieren bzw. Speichern von Diagrammen in Bilder im \gls{jpg}-Format
	\item Das Programm muss in Java 1.6 geschrieben sein
\end{itemize}

\subsection{Wunschkriterien} 

\subsubsection{Hohe Priorität}

	\begin{itemize}
		\item Internationalisierungsmechanismen vorbereiten
		\item Anzeige der Bildliste einschränkbar, durch Auswahl in der Inhaltsliste
		\item Weitere Ausgabeformate unterstützen 
		\item Einstellung der Größe der Thumbnails, in der Projektansicht, mittels eines Schiebereglers
	\end{itemize}

\subsubsection{Mittlere Priorität}

	\begin{itemize}
		\item Optimierung von Algorithmen
		\item Anzeige von Thumbnails, sowie Dateinamen in Diagrammen über eine Mengenauswahl
		\item Vernünftige anpassbare Diagrammskalierungen
		\item Konfigurierbarkeit des Layouts	
	\end{itemize}

\subsubsection{Niedrige Priorität}

	\begin{itemize}
		\item Unterstützen weiterer \gls{exif}-Parameter, sowie kameraspezifischer Parameter
		\item Normierung von Werten, z.B. Brennweitenkorrektur
		\item Unterstützung weiterer Bildformate mit Metadaten 	
		\item Unterstützen weiterer Diagrammtypen
	\end{itemize}

\subsection{Abgrenzungskriterien} 
\begin{itemize}
	\item \gls{tempX} soll keine \gls{exif}-Parameter bearbeiten können
	\item \gls{tempX} soll keine Bilder bearbeiten bzw. löschen können
	\item \gls{tempX} soll keine Bilder ausdrucken können
	\item \gls{tempX} soll keine Diashow anzeigen können
	\item \gls{tempX} muss keinen hohen Sicherheitsansprüchen genügen
\end{itemize}