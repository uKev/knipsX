\subsection{Anwendungsfälle}

	\subsubsection{Programmmanagement:}
	
	\begin{description}
	\item[Anwendungsfall 1]
	\end{description}
	
	\begin{itemize}
		\item Name: Programm starten
		\item Teilnehmende Akteure:
		\begin{itemize}
			\item	Fotograf A.: Benutzer des Programms.
		\end{itemize}
		\item Eingangsbedingung:
		\begin{itemize}
			\item Fotograf A. besitzt das Programm.
			\item Fotograf A. hat das Programm ordnungsgemäß auf seinem PC installiert und eingerichtet.						
		\end{itemize}
		\item Ausgangsbedingung:
		\begin{itemize}
			\item	Fotograf A. hat das Programm gestartet. Es erschein das Projektübersichtsfenster.		
		\end{itemize}
		\item Ereignisfluss:	
		\begin{itemize}
			\item Fotograf A. startet das Programm mit der ausführbaren Datei.		
			\item Das Projektübersichtsfenster wird angezeigt.
		\end{itemize}
		\item Spezielle Anforderungen:
		\begin{itemize}
			\item	Der Computer muss den gegebenen Anforderungen genügen.
		\end{itemize}
	\end{itemize}
	
	\begin{description}
	\item[Anwendungsfall 2]
	\end{description}
	
	\begin{itemize}
		\item Name: Programm schließen
		\item Teilnehmende Akteure:
		\begin{itemize}
			\item	Fotograf A.: Benutzer des Programms.
		\end{itemize}
		\item Eingangsbedingung:
		\begin{itemize}
			\item Fotograf A. hat das Programm geöffnet.
			\item Fotograf A. ist fertig mit seiner Arbeit und will das Programm beenden.						
		\end{itemize}
		\item Ausgangsbedingung:
		\begin{itemize}
			\item	Fotograf A. hat das Programm beendet.		
		\end{itemize}
		\item Ereignisfluss:\\Erste Möglichkeit:	
		\begin{itemize}
			\item Fotograf A. befindet sich im Projektansichtsfenster und klickt oben rechts auf Fenster schließen.
			\item Es kommt ein Fenster mit einer Sicherheitsabfrage auf, welche zum Speichern auffordert.
			\item Fotograf A. bestätigt das Speichern.
			\item Fotograf A. hat somit das Programm beendet. Es verschwindet vom Desktop und aus den laufenden Prozessen.
		\end{itemize}
		Zweite Möglichkeit:
		\begin{itemize}
			\item Fotograf A. befindet sich nicht im Projektansichtsfenster. Daher muss er zuerst ins Projektansichtsfenster zurückkehren, indem er entweder den aktuelle Ansicht schließt oder abbricht.
			\item Fotograf A. befindet sich im Projektansichtsfenster und klickt oben rechts auf Fenster schließen.
			\item Es kommt ein Fenster mit einer Sicherheitsabfrage auf, welche zum Speichern auffordert.
			\item Fotograf A. bestätigt das Speichern.
			\item Fotograf A. hat somit das Programm beendet. Es verschwindet vom Desktop und aus den laufenden Prozessen.
		\end{itemize}	
		\item Spezielle Anforderungen:
		\begin{itemize}
			\item	Alle Eingaben müssen gültig sein.		
		\end{itemize}
	\end{itemize}
	
\begin{description}
	\item[Anwendungsfall 3]
	\end{description}
	
	\begin{itemize}
		\item Name: Programmfenster anpassen
		\item Teilnehmende Akteure:
		\begin{itemize}
			\item	Fotograf A.: Benutzer des Programms.
		\end{itemize}
		\item Eingangsbedingung:
		\begin{itemize}
			\item Fotograf A. hat das Programm geöffnet.
			\item Fotograf A. will sein Programmfenster anpassen.						
		\end{itemize}
		\item Ausgangsbedingung:
		\begin{itemize}
			\item	Fotograf A. hat das Programmfenster seinen Bedürfnissen angepasst.		
		\end{itemize}
		\item Ereignisfluss:	
		\begin{itemize}
			\item Fotograf A. kann das Programmfenster minimieren und maximieren.
			\item Fotograf A. kann das Programmfenster auf der Desktopoberfläche verschieben und positionieren.
			\item Fotograf A. kann das Programmfenster in der Höhe und Breite anpassen, indem er den Rahmen mit der Maus zieht.
			\item Wenn Fotograf A. sein Programmfenster ausgerichtet hat, kann er die Arbeit fortsetzen.
		\end{itemize}
		\item Spezielle Anforderungen:
		\begin{itemize}
			\item	Es muss die Mindestauflösung eingehalten werden.
			\item Die Funktionalität beschränkt sich jeweils auf die Darstellbarkeit auf dem Desktop und dem Bildschirm.
		\end{itemize}
	\end{itemize}	
	
	\subsubsection{Projektmanagement:}
		
	\begin{description}
		\item[Anwendungsfall 4]
	\end{description}
	
		\begin{itemize}
			\item Name: Erstellen eines neuen Projekts
			\item Teilnehmende Akteure:
			\begin{itemize}
				\item	Fotograf A.: Benutzer des Programms.
			\end{itemize}
			\item Eingangsbedingung:
			\begin{itemize}
				\item Fotograf A. will Bilder analysieren bzw. eine Auswertung erstellen.						
			\end{itemize}
			\item Ausgangsbedingung:
			\begin{itemize}
				\item	Fotograf A. hat ein Projekt, mit welchem er arbeiten kann.		
			\end{itemize}
			\item Ereignisfluss:
			\begin{itemize}
				\item Fotograf A. startet das Programm auf seinem Computer.
				\item Fotograf A. bekommt das Projektübersichtsfenster angezeigt. Es befindet sich entweder noch kein Projekt in der Liste oder es sind bereits Projekte vorhanden.
				\item Fotograf A. klickt auf Neues Projekt erstellen.
				\item Es erscheint ein Fenster mit Textfeld.
				\item Fotograf A. gibt einen gültigen Namen für sein Projekt ein.
				\item Fotograf A. bestätigt seine Eingabe.
				\item Fotograf A. gelangt in das Projektansichtsfenster seines Projekts und kann mit seinem Vorhaben beginnen. Ihm wird der Projektname inklusive Bearbeitungsdatum links oben angezeigt.
				\item Der Name des Projekts ist mit letztem Speicherdatum auch in der Liste der Projekte mit aufgenommen.
			\end{itemize}
			\item Spezielle Anforderungen:
			\begin{itemize}
				\item	Das Programm muss korrekt auf dem PC eingerichtet sein.
				\item Alle Eingaben müssen korrekt sein.
			\end{itemize}			
		\end{itemize}
		
	\begin{description}
		\item[Anwendungsfall 5]
	\end{description}
	
		\begin{itemize}
			\item Name: Entfernen eines Projekts
			\item Teilnehmende Akteure:
			\begin{itemize}
				\item	Fotograf A.: Benutzer des Programms.	
			\end{itemize}
			\item Eingangsbedingung:
			\begin{itemize}
				\item	Das Programm befindet sich im Projektübersichtsfenster.		
			\end{itemize}
			\item Ausgangsbedingung:
			\begin{itemize}
				\item	Ein ausgewähltes Projekt wird aus dem Programm und vom Computer entfernt.		
			\end{itemize}
			\item Ereignisfluss:
			\begin{itemize}
				\item Fotograf A. klickt auf das Projekt welches er entfernen will um es zu markieren.
				\item Fotograf A. klickt auf den Button "`Projekt entfernen"'.
				\item Es kommt ein Fenster auf welches eine Sicherheitsabfrage darstellt.
				\item Fotograf A. bestätigt Sicherheitsabfrage mit Ja.
				\item Das Projekt verschwindet aus der Liste.
			\end{itemize}
			\item Spezielle Anforderungen:
			\begin{itemize}
				\item	Es existiert mindestens ein Projekt.		
			\end{itemize}			
		\end{itemize}
		
		\begin{description}
			\item[Anwendungsfall 6]
		\end{description}
		
		\begin{itemize}
			\item Name: Öffnen eines Projekts
			\item Teilnehmende Akteure:
			\begin{itemize}
				\item	Fotograf A.: Benutzer des Programms.		
			\end{itemize}
			\item Eingangsbedingung:
			\begin{itemize}
				\item	Das Programm befindet sich im Projektübersichtsfenster.
				\item Es ist bereits mindestens ein Projekt in der Projektliste vorhanden.
			\end{itemize}
			\item Ausgangsbedingung:
			\begin{itemize}
				\item	Ein bereits erstelltes Projekt ist vollständig geladen und wird Fotograf A. angezeigt. Es kann bearbeitet werden.		
			\end{itemize}
			\item Ereignisfluss:\\Erste Möglichkeit:
			\begin{itemize}
				\item Fotograf A. klickt einmal auf das zu öffnende Projekt um es zu markieren.
				\item Fotograf A. klickt auf Button Projekt öffnen um zum Projekt zu gelangen.
				\item Das Projekt wird im Projektansichtsfenster angezeigt.
			\end{itemize}
			Zweite Möglichkeit:
			\begin{itemize}
				\item Fotograf A. klickt per Doppelklick direkt aud Projektnamen um es direkt zu öffnen.
				\item Das Projekt wird im Projektansichtsfenster angezeigt.					
			\end{itemize}
			\item Spezielle Anforderungen:
			\begin{itemize}
				\item	Es existiert bereits mindestens ein Projekt.
				\item Alle Eingaben sind korrekt.
			\end{itemize}			
		\end{itemize}
		
	\begin{description}
		\item[Anwendungsfall 7]
	\end{description}
	
		\begin{itemize}
			\item Name: Kopieren eines Projekts
			\item Teilnehmende Akteure:
			\begin{itemize}
				\item	Fotograf A.: Benutzer des Programms		
			\end{itemize}
			\item Eingangsbedingung:
			\begin{itemize}
				\item	Das Programm befindet sich im Projektübersichtsfenster.
				\item Es ist bereits mindestens ein Projekt in der Projektliste vorhanden.			
			\end{itemize}
			\item Ausgangsbedingung:
			\begin{itemize}
				\item	Es wurde ein neues Projekt erstellt, welches die selben Eigenschaften und Daten enthält, wie ein anderes.	
			\end{itemize}
			\item Ereignisfluss:
			\begin{itemize}
				\item Fotograf A. klickt einmal auf das zu kopierende Projekt um es zu markieren.
				\item Fotograf A. klickt auf den Button "`Projekt kopieren"'.
				\item Es erscheint ein Fenster mit Textfeld.
				\item Fotograf A. gibt einen gültigen Namen für sein Projekt ein.
				\item Fotograf A. bestätigt seine Eingabe.
				\item Fotograf A. gelangt in das Projektansichtsfenster seines eben erstellten Projekts und kann mit seiner Arbeit beginnen. Ihm wird der Projektname mit Bearbeitungsdatum links oben angezeigt.
				\item Alle Werte und Einstellungen werden vom Originalobjekt übernommen und auch dementsprechend angezeigt.
			\end{itemize}
			\item Spezielle Anforderungen:
			\begin{itemize}
				\item	Es existiert bereits mindestens ein Projekt.		
			\end{itemize}			
		\end{itemize}
		
		\begin{description}
		\item[Anwendungsfall 8]
	\end{description}
	
		\begin{itemize}
			\item Name: Projekts wechseln
			\item Teilnehmender Akteure:
			\begin{itemize}
				\item	Fotograf A.: Benutzer des Programms		
			\end{itemize}
			\item Eingangsbedingung:
			\begin{itemize}
				\item	Das Programm befindet sich im Projektansichtsfenster eines Projekts.
				\item Es ist mindestens ein anderes als das aktuelle Projekt vorhanden.			
			\end{itemize}
			\item Ausgangsbedingung:
			\begin{itemize}
				\item	Fotograf A. hat das Projekt gewechselt. Er arbeitet nun in einem anderen Projekt. Das alte Projekt wurde gespeichert.
			\end{itemize}
			\item Ereignisfluss:
			\begin{itemize}
			  \item Fotograf A. klickt einmal auf "`Projekt speichern"' um das aktuelle Projekt zu speichern.
				\item Fotograf A. klickt einmal auf "`Projekt wechseln"'.
				\item Fotograf A. erhält nochmals eine Sicherheitsabfrage ob er das aktuelle Projekt speichern will.
				\item Fotograf A. bestätigt mit "`Ja"'.
				\item Fotograf A. gelangt in das Projektübersichtsfenster.
				\item Dort kann er so wie in "`Anwendungsfall 6"' beschrieben fortfahren und wählt ein anderes Projekt aus.
			\end{itemize}
			\item Spezielle Anforderungen:
			\begin{itemize}
				\item	Es existiert bereits mindestens ein anderes Projekt.
				\item Alle Eingaben sind korrekt.
			\end{itemize}			
		\end{itemize}
		
	\begin{description}
\item[Anwendungsfall 8.5]
\end{description}
 
\begin{itemize}
\item Name: Projekt umbenennen
\item Teilnehmender Akteure:
\begin{itemize}
\item Fotograf A.: Benutzer des Programms
\end{itemize}
\item Eingangsbedingung:
\begin{itemize}
\item Das Programm befindet sich im Projektansichtsfenster eines Projekts.
\item Fotograf A. will dem aktuellen Projekt einen anderen Namen zuweisen.
\end{itemize}
\item Ausgangsbedingung:
\begin{itemize}
\item Das Projekt hat einen anderen Namen.
\end{itemize}
\item Ereignisfluss:
\begin{itemize}
\item Fotograf A. klickt auf das Textfeld, in dem der Namen des Projekts steht.
\item Fotograf A. editiert das Textfeld nach seinen Bedürfnissen.
\item Fotograf A. speichert das aktuelle Projekt. Nun wird überall der Name verändert angezeigt.
\end{itemize}
\item Spezielle Anforderungen:
\begin{itemize}
\item Es existiert bereits mindestens ein Projekt welches aktiv ist.
\item Alle Eingaben müssen korrekt sein.
\end{itemize}
\end{itemize}

	\subsubsection{Bildmengenmanagement:}
	
	\begin{description}
		\item[Anwendungsfall 9]
	\end{description}
	
	\begin{itemize}
		\item Name: Erstellen einer Bildmenge
		\item Teilnehmende Akteure:
		\begin{itemize}
			\item	Fotograf A.: Benutzer des Programms		
		\end{itemize}
		\item Eingangsbedingung:
		\begin{itemize}
			\item	Das Programm befindet sich im Projektansichtsfenster eines Projekts.
			\item Fotograf A. will eine neue Bildmenge erstellen.
		\end{itemize}
		\item Ausgangsbedingung:
		\begin{itemize}
			\item	Es wurde eine neue Bildmenge erstellt.	
		\end{itemize}
		\item Ereignisfluss:
		\begin{itemize}
			\item Fotograf A. klickt im Projektansichtsfenster im Bereich der Bildmengen auf "`Erstellen"'.		
			\item Nun erscheint ein neues Fenster, in dem ein Namen in einem Textfeld eingetragen werden kann.
			\item Fotograf A. gibt einen gültigen Namen für seine Bildermenge ein.
			\item Fotograf A. bestätigt seine Eingabe.
			\item Fotograf A. gelangt in das Projektansichtsfenster zurück. Seine soeben angelegte Bildmenge wird ihm in der Liste der Bildmengen mit Namen angezeigt.
		\end{itemize}
		\item Spezielle Anforderungen:
		\begin{itemize}
			\item	Es finden nur gültige Eingaben statt.		
		\end{itemize}			
	\end{itemize}
	
	\begin{description}
		\item[Anwendungsfall 10]
	\end{description}
	
	\begin{itemize}
		\item Name: Entfernen einer Bildmenge
		\item Teilnehmende Akteure:
		\begin{itemize}
			\item	Fotograf A.: Benutzer des Programms		
		\end{itemize}
		\item Eingangsbedingung:
		\begin{itemize}
			\item	Das Programm befindet sich im Projektansichtsfenster eines Projekts.
			\item Fotograf A. will eine Bildmenge entfernen.
		\end{itemize}
		\item Ausgangsbedingung:
		\begin{itemize}
			\item	Es wurde eine Bildmenge entfernt.	
		\end{itemize}
		\item Ereignisfluss:
		\begin{itemize}
			\item Fotograf A. klickt im Projektansichtsfenster in der Liste der Bildmengen auf diejenige Bildmenge, die er entfernen will, um sie zu markieren.		
			\item Nun klickt Fotograf A. im Projektansichtsfenster im Bereich der Bildmengen auf "`Entfernen"'.
			\item Ihm wird ein Fenster als Sicherheitsabfrage angezeigt.
			\item Fotograf A. bestätigt das Entfernen der Bildmenge ,it "`Ja"'.
			\item Die Bildmenge wird aus der Bildmengenliste entfernt.
			\item Die nun aktive Bildmenge ist die erste der Bildmengenliste, falls vorhanden, sonst sind die Listen nun leer.
			\item Der Inhalt der Inhaltsliste und von der Bilderliste einer Bildmenge wird aktualisiert. 
		\end{itemize}
		\item Spezielle Anforderungen:
		\begin{itemize}
			\item	Es finden nur gültige Eingaben statt.
			\item Es existiert mindestens eine Bildmenge.
		\end{itemize}			
	\end{itemize}
	
	\begin{description}
		\item[Anwendungsfall 11]
	\end{description}
	
	\begin{itemize}
		\item Name: Kopieren einer Bildmenge
		\item Teilnehmende Akteure:
		\begin{itemize}
			\item	Fotograf A.: Benutzer des Programms		
		\end{itemize}
		\item Eingangsbedingung:
		\begin{itemize}
			\item	Das Programm befindet sich im Projektansichtsfenster eines Projekts.
			\item Fotograf A. will eine Bildmenge kopieren.
		\end{itemize}
		\item Ausgangsbedingung:
		\begin{itemize}
			\item	Es wurde eine Kopie einer bereits vorhandenen Bildmenge inklusive Bildinhalt erstellt.	
		\end{itemize}
		\item Ereignisfluss:
		\begin{itemize}
			\item Fotograf A. klickt im Projektansichtsfenster in der Liste der Bildmengen auf diejenige Bildmenge, die er kopieren will, um sie zu markieren.		
			\item Nun klickt Fotograf A. im Projektansichtsfenster im Bereich der Bildmengen auf "`Kopieren"'.
			\item Ihm wird ein Fenster mit Textfeld angezeigt.
			\item Fotograf A. gibt einen gültigen Namen für seine Bildmenge ein.
			\item Fotograf A. bestätigt seine Eingabe.
			\item Die nun aktive Bildmenge ist die soeben erstellte Kopie. Sie zeigt die gleichen Listeninhalte an wie die Ursprungsbildmenge.
		\end{itemize}
		\item Spezielle Anforderungen:
		\begin{itemize}
			\item	Es finden nur gültige Eingaben statt.
			\item Es existiert mindestens eine Bildmenge.
		\end{itemize}			
	\end{itemize}
	
	\begin{description}
		\item[Anwendungsfall 12]
	\end{description}
	
	\begin{itemize}
		\item Name: Hinzufügen von Bildern zu Bildmengen
		\item Teilnehmende Akteure:
		\begin{itemize}
			\item	Fotograf A.: Benutzer des Programms		
		\end{itemize}
		\item Eingangsbedingung:
		\begin{itemize}
			\item	Das Programm befindet sich im Projektansichtsfenster eines Projekts.
			\item Fotograf A. will einer Bildmenge Bilder zuweisen.
		\end{itemize}
		\item Ausgangsbedingung:
		\begin{itemize}
			\item	Es wurden einer bereits vorhandenen Bildmenge Bilder zugewiesen.
			\item Die zugewiesenen Bilder werden angezeigt.
		\end{itemize}
		\item Ereignisfluss:
		\begin{itemize}
			\item Fotograf A. klickt im Projektansichtsfenster in der Liste der Bildmengen auf diejenige Bildmenge, der er Bilder hinzufügen will, um sie zu markieren.		
			\item Nun klickt Fotograf A. im Projektansichtsfenster im Bereich Inhalt auf "`Hinzufügen"'.
			\item Ihm wird ein Fenster mit Filemanager angezeigt.\\\\Erste Möglichkeit:\\
			\item Fotograf A. wählt entweder ein oder mehrere Ordner mit dem Filemanager aus und drückt auf "`Hinzufügen"'.
			\item Das Projektansichtsfenster wird angezeigt.
			\item Die soeben hinzugefügten Ordner werden mit blauer Schrift im Inhaltsfenster der Bildmenge angezeigt, der sie zugewiesen wurden.
			\item Die Bilderliste einer Bildmenge wird mit allen Bildern aktualisiert die sich nun in der Bildmenge befinden.
			\item Der Bildstatus aller Bilder ist nach dem Hinzufügen immer aktiv.\\\\Zweite Möglichkeit:\\
			\item Fotograf A. will in einen Ordner schauen und kann diesen mit einem Doppelklick öffnen.
			\item Fotograf A. klickt auf ein Bild um es zu markieren, er bekommt dabei eine Vorschau des Bildes angezeigt.
			\item Fotograf A. wählt entweder ein oder mehrere Bilder mit dem Filemanager aus und drückt dann auf "`Hinzufügen"'.		
			\item Das Projektansichtsfenster wird angezeigt.
			\item Die soeben hinzugefügten Bilder werden mit schwarzer Schrift im Inhaltsfenster der Bildmenge angezeigt, der sie zugewiesen wurden.
			\item Die Bilderliste einer Bildmenge wird mit allen Bildern aktualisiert die sich nun in der Bildmenge befinden.
			\item Der Bildstatus aller Bilder ist nach dem Hinzufügen immer aktiv.
		\end{itemize}
		\item Spezielle Anforderungen:
		\begin{itemize}
			\item	Es finden nur gültige Eingaben statt.
			\item Es existiert mindestens eine Bildmenge.
			\item Es existieren Bilder und Ordner mit Bildern als Inhalt auf dem Computer.
			\item Alle Verzeichnispfade sind gültig.
		\end{itemize}			
	\end{itemize}
	
	\begin{description}
		\item[Anwendungsfall 13]
	\end{description}
	
	\begin{itemize}
		\item Name: Einfügen von Bildmengen zu Bildmengen
		\item Teilnehmende Akteure:
		\begin{itemize}
			\item	Fotograf A.: Benutzer des Programms		
		\end{itemize}
		\item Eingangsbedingung:
		\begin{itemize}
			\item	Das Programm befindet sich im Projektansichtsfenster eines Projekts.
			\item Fotograf A. will eine Bildmenge zu einer anderen Bildmenge hinzufügen.
		\end{itemize}
		\item Ausgangsbedingung:
		\begin{itemize}
			\item	Eine Bildmenge hat nun als Inhalt bzw. Teilmenge eine andere Bildmenge mit deren Inhalt. 	
		\end{itemize}
		\item Ereignisfluss:
		\begin{itemize}
			\item Fotograf A. klickt im Projektansichtsfenster in der Liste der Bildmengen auf diejenige Bildmenge, in die eingefügt werden soll, um sie zu markieren.
			\item Diese Bildmenge wird nun in allen Fenstern geladen und dargestellt.
			\item Nun klickt Fotograf A. im Projektansichtsfenster im Bereich der Bildmengen auf diejenige Bildmenge, die er einfügen will und lässt dabei die Maus nicht los. Er kann diese dann nach unten in das Inhaltsfenster der anderen Bildmenge ziehen. Befindet sich die Bildmenge über dem Inhaltfenster kann er die Maus loslassen.
			\item Nun erscheint die Bildmenge in grüner Schrift im Inhaltsfenster. Dabei bleibt sie aber in der bildmengenliste stehen.
			\item Die Bilderliste einer Bildmenge wird mit allen Bildern aktualisiert die sich nun in der Bildmenge befinden.
		\end{itemize}
		\item Spezielle Anforderungen:
		\begin{itemize}
			\item	Es finden nur gültige Eingaben statt.
			\item Es existieren mindestens zwei Bildmengen.
		\end{itemize}			
	\end{itemize}
	
	\begin{description}
		\item[Anwendungsfall 14]
	\end{description}
	
	\begin{itemize}
		\item Name: Das Entfernen von Inhalten der Bildmengen
		\item Teilnehmende Akteure:
		\begin{itemize}
			\item	Fotograf A.: Benutzer des Programms		
		\end{itemize}
		\item Eingangsbedingung:
		\begin{itemize}
			\item	Das Programm befindet sich im Projektansichtsfenster eines Projekts.
			\item Fotograf A. will einen oder mehrere Bildmengeninhalte entfernen.
		\end{itemize}
		\item Ausgangsbedingung:
		\begin{itemize}
			\item	Es wurden Inhalte aus Bildmengen entfernt.	
		\end{itemize}
		\item Ereignisfluss:
		\begin{itemize}
			\item Fotograf A. klickt im Projektansichtsfenster in der Liste der Bildmengen auf diejenige Bildmenge, in der etwas entfernt werden soll, um sie zu markieren.		
			\item Diese Bildmenge wird nun in allen Fenstern geladen und dargestellt.
			\item Fotograf A. klickt im Projektansichtsfenster in der Liste des Inhalts der aktuellen Bildmenge auf eine oder mehrere Bilddateien, Ordner oder Bildmengen und aktiviert bzw. markiert diese.
			\item Fotograf A. drückt dannach den "`Entfernen Button"' welcher zur Liste gehört.
			\item Im folgenden verschwinden alle Einträge, welche zu entfernen waren, aus der Inhaltsliste.
			\item Die Bilderliste einer Bildmenge wird mit allen Bildern aktualisiert die sich nun in der Bildmenge befinden.
		\end{itemize}
		\item Spezielle Anforderungen:
		\begin{itemize}
			\item	Es finden nur gültige Eingaben statt.
			\item Es existiert mindestens eine Bildmenge mit Inhalt.
			\item Alle Verzeichnispfade sind gültig.
		\end{itemize}			
	\end{itemize}
	
	\begin{description}
		\item[Anwendungsfall 15]
	\end{description}
	
	\begin{itemize}
		\item Name: Aktualisieren der Listeninhalte
		\item Teilnehmende Akteure:
		\begin{itemize}
			\item	Fotograf A.: Benutzer des Programms		
		\end{itemize}
		\item Eingangsbedingung:
		\begin{itemize}
			\item	Das Programm befindet sich im Projektansichtsfenster eines Projekts.
			\item Fotograf A. hat außerhalb des Programms Änderungen am Dateisystem vorgenommen und Daten sowie Verzeichnisse manipuliert.
		\end{itemize}
		\item Ausgangsbedingung:
		\begin{itemize}
			\item	Es werden aktuelle Inhalte angezeigt.	
		\end{itemize}
		\end{itemize}
		
\subsection{Anwendungsfälle}

	\subsubsection{Programmmanagement:}
	
	\begin{description}
	\item[Anwendungsfall 1]
	\end{description}
	
	\begin{itemize}
		\item Name: Programm starten
		\item Teilnehmende Akteure:
		\begin{itemize}
			\item	Fotograf A.: Benutzer des Programms.
		\end{itemize}
		\item Eingangsbedingung:
		\begin{itemize}
			\item Fotograf A. besitzt das Programm.
			\item Fotograf A. hat das Programm ordnungsgemäß auf seinem PC installiert und eingerichtet.						
		\end{itemize}
		\item Ausgangsbedingung:
		\begin{itemize}
			\item	Fotograf A. hat das Programm gestartet. Es erschein das Projektübersichtsfenster.		
		\end{itemize}
		\item Ereignisfluss:	
		\begin{itemize}
			\item Fotograf A. startet das Programm mit der ausführbaren Datei.		
			\item Das Projektübersichtsfenster wird angezeigt.
		\end{itemize}
		\item Spezielle Anforderungen:
		\begin{itemize}
			\item	Der Computer muss den gegebenen Anforderungen genügen.
		\end{itemize}
	\end{itemize}
	
	\begin{description}
	\item[Anwendungsfall 2]
	\end{description}
	
	\begin{itemize}
		\item Name: Programm schließen
		\item Teilnehmende Akteure:
		\begin{itemize}
			\item	Fotograf A.: Benutzer des Programms.
		\end{itemize}
		\item Eingangsbedingung:
		\begin{itemize}
			\item Fotograf A. hat das Programm geöffnet.
			\item Fotograf A. ist fertig mit seiner Arbeit und will das Programm beenden.						
		\end{itemize}
		\item Ausgangsbedingung:
		\begin{itemize}
			\item	Fotograf A. hat das Programm beendet.		
		\end{itemize}
		\item Ereignisfluss:\\Erste Möglichkeit:	
		\begin{itemize}
			\item Fotograf A. befindet sich im Projektansichtsfenster und klickt oben rechts auf Fenster schließen.
			\item Es kommt ein Fenster mit einer Sicherheitsabfrage auf, welche zum Speichern auffordert.
			\item Fotograf A. bestätigt das Speichern.
			\item Fotograf A. hat somit das Programm beendet. Es verschwindet vom Desktop und aus den laufenden Prozessen.
		\end{itemize}
		Zweite Möglichkeit:
		\begin{itemize}
			\item Fotograf A. befindet sich nicht im Projektansichtsfenster. Daher muss er zuerst ins Projektansichtsfenster zurückkehren, indem er entweder den aktuelle Ansicht schließt oder abbricht.
			\item Fotograf A. befindet sich im Projektansichtsfenster und klickt oben rechts auf Fenster schließen.
			\item Es kommt ein Fenster mit einer Sicherheitsabfrage auf, welche zum Speichern auffordert.
			\item Fotograf A. bestätigt das Speichern.
			\item Fotograf A. hat somit das Programm beendet. Es verschwindet vom Desktop und aus den laufenden Prozessen.
		\end{itemize}	
		\item Spezielle Anforderungen:
		\begin{itemize}
			\item	Alle Eingaben müssen gültig sein.		
		\end{itemize}
	\end{itemize}
	
\begin{description}
	\item[Anwendungsfall 3]
	\end{description}
	
	\begin{itemize}
		\item Name: Programmfenster anpassen
		\item Teilnehmende Akteure:
		\begin{itemize}
			\item	Fotograf A.: Benutzer des Programms.
		\end{itemize}
		\item Eingangsbedingung:
		\begin{itemize}
			\item Fotograf A. hat das Programm geöffnet.
			\item Fotograf A. will sein Programmfenster anpassen.						
		\end{itemize}
		\item Ausgangsbedingung:
		\begin{itemize}
			\item	Fotograf A. hat das Programmfenster seinen Bedürfnissen angepasst.		
		\end{itemize}
		\item Ereignisfluss:	
		\begin{itemize}
			\item Fotograf A. kann das Programmfenster minimieren und maximieren.
			\item Fotograf A. kann das Programmfenster auf der Desktopoberfläche verschieben und positionieren.
			\item Fotograf A. kann das Programmfenster in der Höhe und Breite anpassen, indem er den Rahmen mit der Maus zieht.
			\item Wenn Fotograf A. sein Programmfenster ausgerichtet hat, kann er die Arbeit fortsetzen.
		\end{itemize}
		\item Spezielle Anforderungen:
		\begin{itemize}
			\item	Es muss die Mindestauflösung eingehalten werden.
			\item Die Funktionalität beschränkt sich jeweils auf die Darstellbarkeit auf dem Desktop und dem Bildschirm.
		\end{itemize}
	\end{itemize}	
	
	\subsubsection{Projektmanagement:}
		
	\begin{description}
		\item[Anwendungsfall 4]
	\end{description}
	
		\begin{itemize}
			\item Name: Erstellen eines neuen Projekts
			\item Teilnehmende Akteure:
			\begin{itemize}
				\item	Fotograf A.: Benutzer des Programms.
			\end{itemize}
			\item Eingangsbedingung:
			\begin{itemize}
				\item Fotograf A. will Bilder analysieren bzw. eine Auswertung erstellen.						
			\end{itemize}
			\item Ausgangsbedingung:
			\begin{itemize}
				\item	Fotograf A. hat ein Projekt, mit welchem er arbeiten kann.		
			\end{itemize}
			\item Ereignisfluss:
			\begin{itemize}
				\item Fotograf A. startet das Programm auf seinem Computer.
				\item Fotograf A. bekommt das Projektübersichtsfenster angezeigt. Es befindet sich entweder noch kein Projekt in der Liste oder es sind bereits Projekte vorhanden.
				\item Fotograf A. klickt auf Neues Projekt erstellen.
				\item Es erscheint ein Fenster mit Textfeld.
				\item Fotograf A. gibt einen gültigen Namen für sein Projekt ein.
				\item Fotograf A. bestätigt seine Eingabe.
				\item Fotograf A. gelangt in das Projektansichtsfenster seines Projekts und kann mit seinem Vorhaben beginnen. Ihm wird der Projektname inklusive Bearbeitungsdatum links oben angezeigt.
				\item Der Name des Projekts ist mit letztem Speicherdatum auch in der Liste der Projekte mit aufgenommen.
			\end{itemize}
			\item Spezielle Anforderungen:
			\begin{itemize}
				\item	Das Programm muss korrekt auf dem PC eingerichtet sein.
				\item Alle Eingaben müssen korrekt sein.
			\end{itemize}			
		\end{itemize}
		
	\begin{description}
		\item[Anwendungsfall 5]
	\end{description}
	
		\begin{itemize}
			\item Name: Entfernen eines Projekts
			\item Teilnehmende Akteure:
			\begin{itemize}
				\item	Fotograf A.: Benutzer des Programms.	
			\end{itemize}
			\item Eingangsbedingung:
			\begin{itemize}
				\item	Das Programm befindet sich im Projektübersichtsfenster.		
			\end{itemize}
			\item Ausgangsbedingung:
			\begin{itemize}
				\item	Ein ausgewähltes Projekt wird aus dem Programm und vom Computer entfernt.		
			\end{itemize}
			\item Ereignisfluss:
			\begin{itemize}
				\item Fotograf A. klickt auf das Projekt welches er entfernen will um es zu markieren.
				\item Fotograf A. klickt auf den Button "`Projekt entfernen"'.
				\item Es kommt ein Fenster auf welches eine Sicherheitsabfrage darstellt.
				\item Fotograf A. bestätigt Sicherheitsabfrage mit Ja.
				\item Das Projekt verschwindet aus der Liste.
			\end{itemize}
			\item Spezielle Anforderungen:
			\begin{itemize}
				\item	Es existiert mindestens ein Projekt.		
			\end{itemize}			
		\end{itemize}
		
		\begin{description}
			\item[Anwendungsfall 6]
		\end{description}
		
		\begin{itemize}
			\item Name: Öffnen eines Projekts
			\item Teilnehmende Akteure:
			\begin{itemize}
				\item	Fotograf A.: Benutzer des Programms.		
			\end{itemize}
			\item Eingangsbedingung:
			\begin{itemize}
				\item	Das Programm befindet sich im Projektübersichtsfenster.
				\item Es ist bereits mindestens ein Projekt in der Projektliste vorhanden.
			\end{itemize}
			\item Ausgangsbedingung:
			\begin{itemize}
				\item	Ein bereits erstelltes Projekt ist vollständig geladen und wird Fotograf A. angezeigt. Es kann bearbeitet werden.		
			\end{itemize}
			\item Ereignisfluss:\\Erste Möglichkeit:
			\begin{itemize}
				\item Fotograf A. klickt einmal auf das zu öffnende Projekt um es zu markieren.
				\item Fotograf A. klickt auf Button Projekt öffnen um zum Projekt zu gelangen.
				\item Das Projekt wird im Projektansichtsfenster angezeigt.
			\end{itemize}
			Zweite Möglichkeit:
			\begin{itemize}
				\item Fotograf A. klickt per Doppelklick direkt aud Projektnamen um es direkt zu öffnen.
				\item Das Projekt wird im Projektansichtsfenster angezeigt.					
			\end{itemize}
			\item Spezielle Anforderungen:
			\begin{itemize}
				\item	Es existiert bereits mindestens ein Projekt.
				\item Alle Eingaben sind korrekt.
			\end{itemize}			
		\end{itemize}
		
	\begin{description}
		\item[Anwendungsfall 7]
	\end{description}
	
		\begin{itemize}
			\item Name: Kopieren eines Projekts
			\item Teilnehmende Akteure:
			\begin{itemize}
				\item	Fotograf A.: Benutzer des Programms		
			\end{itemize}
			\item Eingangsbedingung:
			\begin{itemize}
				\item	Das Programm befindet sich im Projektübersichtsfenster.
				\item Es ist bereits mindestens ein Projekt in der Projektliste vorhanden.			
			\end{itemize}
			\item Ausgangsbedingung:
			\begin{itemize}
				\item	Es wurde ein neues Projekt erstellt, welches die selben Eigenschaften und Daten enthält, wie ein anderes.	
			\end{itemize}
			\item Ereignisfluss:
			\begin{itemize}
				\item Fotograf A. klickt einmal auf das zu kopierende Projekt um es zu markieren.
				\item Fotograf A. klickt auf den Button "`Projekt kopieren"'.
				\item Es erscheint ein Fenster mit Textfeld.
				\item Fotograf A. gibt einen gültigen Namen für sein Projekt ein.
				\item Fotograf A. bestätigt seine Eingabe.
				\item Fotograf A. gelangt in das Projektansichtsfenster seines eben erstellten Projekts und kann mit seiner Arbeit beginnen. Ihm wird der Projektname mit Bearbeitungsdatum links oben angezeigt.
				\item Alle Werte und Einstellungen werden vom Originalobjekt übernommen und auch dementsprechend angezeigt.
			\end{itemize}
			\item Spezielle Anforderungen:
			\begin{itemize}
				\item	Es existiert bereits mindestens ein Projekt.		
			\end{itemize}			
		\end{itemize}
		
		\begin{description}
		\item[Anwendungsfall 8]
	\end{description}
	
		\begin{itemize}
			\item Name: Projekts wechseln
			\item Teilnehmender Akteure:
			\begin{itemize}
				\item	Fotograf A.: Benutzer des Programms		
			\end{itemize}
			\item Eingangsbedingung:
			\begin{itemize}
				\item	Das Programm befindet sich im Projektansichtsfenster eines Projekts.
				\item Es ist mindestens ein anderes als das aktuelle Projekt vorhanden.			
			\end{itemize}
			\item Ausgangsbedingung:
			\begin{itemize}
				\item	Fotograf A. hat das Projekt gewechselt. Er arbeitet nun in einem anderen Projekt. Das alte Projekt wurde gespeichert.
			\end{itemize}
			\item Ereignisfluss:
			\begin{itemize}
			  \item Fotograf A. klickt einmal auf "`Projekt speichern"' um das aktuelle Projekt zu speichern.
				\item Fotograf A. klickt einmal auf "`Projekt wechseln"'.
				\item Fotograf A. erhält nochmals eine Sicherheitsabfrage ob er das aktuelle Projekt speichern will.
				\item Fotograf A. bestätigt mit "`Ja"'.
				\item Fotograf A. gelangt in das Projektübersichtsfenster.
				\item Dort kann er so wie in "`Anwendungsfall 6"' beschrieben fortfahren und wählt ein anderes Projekt aus.
			\end{itemize}
			\item Spezielle Anforderungen:
			\begin{itemize}
				\item	Es existiert bereits mindestens ein anderes Projekt.
				\item Alle Eingaben sind korrekt.
			\end{itemize}			
		\end{itemize}
		
	\begin{description}
\item[Anwendungsfall 8.5]
\end{description}
 
\begin{itemize}
\item Name: Projekt umbenennen
\item Teilnehmender Akteure:
\begin{itemize}
\item Fotograf A.: Benutzer des Programms
\end{itemize}
\item Eingangsbedingung:
\begin{itemize}
\item Das Programm befindet sich im Projektansichtsfenster eines Projekts.
\item Fotograf A. will dem aktuellen Projekt einen anderen Namen zuweisen.
\end{itemize}
\item Ausgangsbedingung:
\begin{itemize}
\item Das Projekt hat einen anderen Namen.
\end{itemize}
\item Ereignisfluss:
\begin{itemize}
\item Fotograf A. klickt auf das Textfeld, in dem der Namen des Projekts steht.
\item Fotograf A. editiert das Textfeld nach seinen Bedürfnissen.
\item Fotograf A. speichert das aktuelle Projekt. Nun wird überall der Name verändert angezeigt.
\end{itemize}
\item Spezielle Anforderungen:
\begin{itemize}
\item Es existiert bereits mindestens ein Projekt welches aktiv ist.
\item Alle Eingaben müssen korrekt sein.
\end{itemize}
\end{itemize}

	\subsubsection{Bildmengenmanagement:}
	
	\begin{description}
		\item[Anwendungsfall 9]
	\end{description}
	
	\begin{itemize}
		\item Name: Erstellen einer Bildmenge
		\item Teilnehmende Akteure:
		\begin{itemize}
			\item	Fotograf A.: Benutzer des Programms		
		\end{itemize}
		\item Eingangsbedingung:
		\begin{itemize}
			\item	Das Programm befindet sich im Projektansichtsfenster eines Projekts.
			\item Fotograf A. will eine neue Bildmenge erstellen.
		\end{itemize}
		\item Ausgangsbedingung:
		\begin{itemize}
			\item	Es wurde eine neue Bildmenge erstellt.	
		\end{itemize}
		\item Ereignisfluss:
		\begin{itemize}
			\item Fotograf A. klickt im Projektansichtsfenster im Bereich der Bildmengen auf "`Erstellen"'.		
			\item Nun erscheint ein neues Fenster, in dem ein Namen in einem Textfeld eingetragen werden kann.
			\item Fotograf A. gibt einen gültigen Namen für seine Bildermenge ein.
			\item Fotograf A. bestätigt seine Eingabe.
			\item Fotograf A. gelangt in das Projektansichtsfenster zurück. Seine soeben angelegte Bildmenge wird ihm in der Liste der Bildmengen mit Namen angezeigt.
		\end{itemize}
		\item Spezielle Anforderungen:
		\begin{itemize}
			\item	Es finden nur gültige Eingaben statt.		
		\end{itemize}			
	\end{itemize}
	
	\begin{description}
		\item[Anwendungsfall 10]
	\end{description}
	
	\begin{itemize}
		\item Name: Entfernen einer Bildmenge
		\item Teilnehmende Akteure:
		\begin{itemize}
			\item	Fotograf A.: Benutzer des Programms		
		\end{itemize}
		\item Eingangsbedingung:
		\begin{itemize}
			\item	Das Programm befindet sich im Projektansichtsfenster eines Projekts.
			\item Fotograf A. will eine Bildmenge entfernen.
		\end{itemize}
		\item Ausgangsbedingung:
		\begin{itemize}
			\item	Es wurde eine Bildmenge entfernt.	
		\end{itemize}
		\item Ereignisfluss:
		\begin{itemize}
			\item Fotograf A. klickt im Projektansichtsfenster in der Liste der Bildmengen auf diejenige Bildmenge, die er entfernen will, um sie zu markieren.		
			\item Nun klickt Fotograf A. im Projektansichtsfenster im Bereich der Bildmengen auf "`Entfernen"'.
			\item Ihm wird ein Fenster als Sicherheitsabfrage angezeigt.
			\item Fotograf A. bestätigt das Entfernen der Bildmenge ,it "`Ja"'.
			\item Die Bildmenge wird aus der Bildmengenliste entfernt.
			\item Die nun aktive Bildmenge ist die erste der Bildmengenliste, falls vorhanden, sonst sind die Listen nun leer.
			\item Der Inhalt der Inhaltsliste und von der Bilderliste einer Bildmenge wird aktualisiert. 
		\end{itemize}
		\item Spezielle Anforderungen:
		\begin{itemize}
			\item	Es finden nur gültige Eingaben statt.
			\item Es existiert mindestens eine Bildmenge.
		\end{itemize}			
	\end{itemize}
	
	\begin{description}
		\item[Anwendungsfall 11]
	\end{description}
	
	\begin{itemize}
		\item Name: Kopieren einer Bildmenge
		\item Teilnehmende Akteure:
		\begin{itemize}
			\item	Fotograf A.: Benutzer des Programms		
		\end{itemize}
		\item Eingangsbedingung:
		\begin{itemize}
			\item	Das Programm befindet sich im Projektansichtsfenster eines Projekts.
			\item Fotograf A. will eine Bildmenge kopieren.
		\end{itemize}
		\item Ausgangsbedingung:
		\begin{itemize}
			\item	Es wurde eine Kopie einer bereits vorhandenen Bildmenge inklusive Bildinhalt erstellt.	
		\end{itemize}
		\item Ereignisfluss:
		\begin{itemize}
			\item Fotograf A. klickt im Projektansichtsfenster in der Liste der Bildmengen auf diejenige Bildmenge, die er kopieren will, um sie zu markieren.		
			\item Nun klickt Fotograf A. im Projektansichtsfenster im Bereich der Bildmengen auf "`Kopieren"'.
			\item Ihm wird ein Fenster mit Textfeld angezeigt.
			\item Fotograf A. gibt einen gültigen Namen für seine Bildmenge ein.
			\item Fotograf A. bestätigt seine Eingabe.
			\item Die nun aktive Bildmenge ist die soeben erstellte Kopie. Sie zeigt die gleichen Listeninhalte an wie die Ursprungsbildmenge.
		\end{itemize}
		\item Spezielle Anforderungen:
		\begin{itemize}
			\item	Es finden nur gültige Eingaben statt.
			\item Es existiert mindestens eine Bildmenge.
		\end{itemize}			
	\end{itemize}
	
	\begin{description}
		\item[Anwendungsfall 12]
	\end{description}
	
	\begin{itemize}
		\item Name: Hinzufügen von Bildern zu Bildmengen
		\item Teilnehmende Akteure:
		\begin{itemize}
			\item	Fotograf A.: Benutzer des Programms		
		\end{itemize}
		\item Eingangsbedingung:
		\begin{itemize}
			\item	Das Programm befindet sich im Projektansichtsfenster eines Projekts.
			\item Fotograf A. will einer Bildmenge Bilder zuweisen.
		\end{itemize}
		\item Ausgangsbedingung:
		\begin{itemize}
			\item	Es wurden einer bereits vorhandenen Bildmenge Bilder zugewiesen.
			\item Die zugewiesenen Bilder werden angezeigt.
		\end{itemize}
		\item Ereignisfluss:
		\begin{itemize}
			\item Fotograf A. klickt im Projektansichtsfenster in der Liste der Bildmengen auf diejenige Bildmenge, der er Bilder hinzufügen will, um sie zu markieren.		
			\item Nun klickt Fotograf A. im Projektansichtsfenster im Bereich Inhalt auf "`Hinzufügen"'.
			\item Ihm wird ein Fenster mit Filemanager angezeigt.\\\\Erste Möglichkeit:\\
			\item Fotograf A. wählt entweder ein oder mehrere Ordner mit dem Filemanager aus und drückt auf "`Hinzufügen"'.
			\item Das Projektansichtsfenster wird angezeigt.
			\item Die soeben hinzugefügten Ordner werden mit blauer Schrift im Inhaltsfenster der Bildmenge angezeigt, der sie zugewiesen wurden.
			\item Die Bilderliste einer Bildmenge wird mit allen Bildern aktualisiert die sich nun in der Bildmenge befinden.
			\item Der Bildstatus aller Bilder ist nach dem Hinzufügen immer aktiv.\\\\Zweite Möglichkeit:\\
			\item Fotograf A. will in einen Ordner schauen und kann diesen mit einem Doppelklick öffnen.
			\item Fotograf A. klickt auf ein Bild um es zu markieren, er bekommt dabei eine Vorschau des Bildes angezeigt.
			\item Fotograf A. wählt entweder ein oder mehrere Bilder mit dem Filemanager aus und drückt dann auf "`Hinzufügen"'.		
			\item Das Projektansichtsfenster wird angezeigt.
			\item Die soeben hinzugefügten Bilder werden mit schwarzer Schrift im Inhaltsfenster der Bildmenge angezeigt, der sie zugewiesen wurden.
			\item Die Bilderliste einer Bildmenge wird mit allen Bildern aktualisiert die sich nun in der Bildmenge befinden.
			\item Der Bildstatus aller Bilder ist nach dem Hinzufügen immer aktiv.
		\end{itemize}
		\item Spezielle Anforderungen:
		\begin{itemize}
			\item	Es finden nur gültige Eingaben statt.
			\item Es existiert mindestens eine Bildmenge.
			\item Es existieren Bilder und Ordner mit Bildern als Inhalt auf dem Computer.
			\item Alle Verzeichnispfade sind gültig.
		\end{itemize}			
	\end{itemize}
	
	\begin{description}
		\item[Anwendungsfall 13]
	\end{description}
	
	\begin{itemize}
		\item Name: Einfügen von Bildmengen zu Bildmengen
		\item Teilnehmende Akteure:
		\begin{itemize}
			\item	Fotograf A.: Benutzer des Programms		
		\end{itemize}
		\item Eingangsbedingung:
		\begin{itemize}
			\item	Das Programm befindet sich im Projektansichtsfenster eines Projekts.
			\item Fotograf A. will eine Bildmenge zu einer anderen Bildmenge hinzufügen.
		\end{itemize}
		\item Ausgangsbedingung:
		\begin{itemize}
			\item	Eine Bildmenge hat nun als Inhalt bzw. Teilmenge eine andere Bildmenge mit deren Inhalt. 	
		\end{itemize}
		\item Ereignisfluss:
		\begin{itemize}
			\item Fotograf A. klickt im Projektansichtsfenster in der Liste der Bildmengen auf diejenige Bildmenge, in die eingefügt werden soll, um sie zu markieren.
			\item Diese Bildmenge wird nun in allen Fenstern geladen und dargestellt.
			\item Nun klickt Fotograf A. im Projektansichtsfenster im Bereich der Bildmengen auf diejenige Bildmenge, die er einfügen will und lässt dabei die Maus nicht los. Er kann diese dann nach unten in das Inhaltsfenster der anderen Bildmenge ziehen. Befindet sich die Bildmenge über dem Inhaltfenster kann er die Maus loslassen.
			\item Nun erscheint die Bildmenge in grüner Schrift im Inhaltsfenster. Dabei bleibt sie aber in der bildmengenliste stehen.
			\item Die Bilderliste einer Bildmenge wird mit allen Bildern aktualisiert die sich nun in der Bildmenge befinden.
		\end{itemize}
		\item Spezielle Anforderungen:
		\begin{itemize}
			\item	Es finden nur gültige Eingaben statt.
			\item Es existieren mindestens zwei Bildmengen.
		\end{itemize}			
	\end{itemize}
	
	\begin{description}
		\item[Anwendungsfall 14]
	\end{description}
	
	\begin{itemize}
		\item Name: Das Entfernen von Inhalten der Bildmengen
		\item Teilnehmende Akteure:
		\begin{itemize}
			\item	Fotograf A.: Benutzer des Programms		
		\end{itemize}
		\item Eingangsbedingung:
		\begin{itemize}
			\item	Das Programm befindet sich im Projektansichtsfenster eines Projekts.
			\item Fotograf A. will einen oder mehrere Bildmengeninhalte entfernen.
		\end{itemize}
		\item Ausgangsbedingung:
		\begin{itemize}
			\item	Es wurden Inhalte aus Bildmengen entfernt.	
		\end{itemize}
		\item Ereignisfluss:
		\begin{itemize}
			\item Fotograf A. klickt im Projektansichtsfenster in der Liste der Bildmengen auf diejenige Bildmenge, in der etwas entfernt werden soll, um sie zu markieren.		
			\item Diese Bildmenge wird nun in allen Fenstern geladen und dargestellt.
			\item Fotograf A. klickt im Projektansichtsfenster in der Liste des Inhalts der aktuellen Bildmenge auf eine oder mehrere Bilddateien, Ordner oder Bildmengen und aktiviert bzw. markiert diese.
			\item Fotograf A. drückt dannach den "`Entfernen Button"' welcher zur Liste gehört.
			\item Im folgenden verschwinden alle Einträge, welche zu entfernen waren, aus der Inhaltsliste.
			\item Die Bilderliste einer Bildmenge wird mit allen Bildern aktualisiert die sich nun in der Bildmenge befinden.
		\end{itemize}
		\item Spezielle Anforderungen:
		\begin{itemize}
			\item	Es finden nur gültige Eingaben statt.
			\item Es existiert mindestens eine Bildmenge mit Inhalt.
			\item Alle Verzeichnispfade sind gültig.
		\end{itemize}			
	\end{itemize}
	
	\begin{description}
		\item[Anwendungsfall 15]
	\end{description}
	
	\begin{itemize}
		\item Name: Aktualisieren der Listeninhalte
		\item Teilnehmende Akteure:
		\begin{itemize}
			\item	Fotograf A.: Benutzer des Programms		
		\end{itemize}
		\item Eingangsbedingung:
		\begin{itemize}
			\item	Das Programm befindet sich im Projektansichtsfenster eines Projekts.
			\item Fotograf A. hat außerhalb des Programms Änderungen am Dateisystem vorgenommen und Daten sowie Verzeichnisse manipuliert.
		\end{itemize}
		\item Ausgangsbedingung:
		\begin{itemize}
			\item	Es werden aktuelle Inhalte angezeigt.	
		\end{itemize}
		\item Ereignisfluss:
		\begin{itemize}
			\item Fotograf A. klickt im Projektansichtsfenster auf den Button "`Aktualisieren"'
			\item Im folgenden werden alle Listen aktualisiert und gegebenfalls verändert angezeigt.
		\end{itemize}
		\item Spezielle Anforderungen:
		\begin{itemize}
			\item	Es finden nur gültige Eingaben statt.
		\end{itemize}			
	\end{itemize}
	
	\begin{description}
		\item[Anwendungsfall 16]
	\end{description}
	
	\begin{itemize}
		\item Name: Aktivieren und Deaktivieren von Bildern
		\item Teilnehmende Akteure:
		\begin{itemize}
			\item	Fotograf A.: Benutzer des Programms		
		\end{itemize}
		\item Eingangsbedingung:
		\begin{itemize}
			\item	Das Programm befindet sich im Projektansichtsfenster eines Projekts.
			\item Fotograf A. hat vor, Bildermengen auszuwerten. Er will nicht alle Bilder einer Bildermenge für die Auswertung berücksichtigen.
		\end{itemize}
		\item Ausgangsbedingung:
		\begin{itemize}
			\item	Es existieren Bildermengen, bei denen nicht alle Bilder berücksichtigt werden.	
		\end{itemize}
		\item Ereignisfluss:
		\begin{itemize}
			\item Fotograf A. klickt im Projektansichtsfenster in der Bilderliste einer Bildmenge ein oder mehrere Bilder an um diese zu aktivieren bzw. deaktivieren.
			\item Bei den jeweiligen Bildern werden Häkchen gesetzt oder entfernt sowie bei den deaktivierten ihre Darstellung verändert.
		\end{itemize}
		\item Spezielle Anforderungen:
		\begin{itemize}
			\item	Es finden nur gültige Eingaben statt.
			\item Es befinden sich Bilder in der aktiven Bildermenge.
		\end{itemize}			
	\end{itemize}
	
	\begin{description}
		\item[Anwendungsfall 17]
	\end{description}
	
	\begin{itemize}
		\item Name: Das Anzeigen einer Bildvorschau
		\item Teilnehmende Akteure:
		\begin{itemize}
			\item	Fotograf A.: Benutzer des Programms		
		\end{itemize}
		\item Eingangsbedingung:
		\begin{itemize}
			\item	Das Programm befindet sich im Projektansichtsfenster eines Projekts.
			\item Fotograf A. hat vor, Bilder in einer Bildmenge zu betrachten.
		\end{itemize}
		\item Ausgangsbedingung:
		\begin{itemize}
			\item	Es wird eine Bildvorschau angezeigt.	
		\end{itemize}
		\item Ereignisfluss:
		\begin{itemize}
			\item Fotograf A. fährt mit dem Mauszeiger im Projektansichtsfenster in der Bilderliste einer Bildmenge über einen Dateinamen und bleibt dort mit dem Mauzeiger stehen.
			\item Es wird das Bild auf welches gezeigt wird in einer gewissen Größe im Vordergrund ausgegeben.
		\end{itemize}
		\item Spezielle Anforderungen:
		\begin{itemize}
			\item	Es finden nur gültige Eingaben statt.
			\item Es befinden sich Bilder in der aktiven Bildermenge.
			\item Die Bilder können angezeigt werden.			
		\end{itemize}			
	\end{itemize}
	
	\begin{description}
		\item[Anwendungsfall 18]
	\end{description}
	
	\begin{itemize}
		\item Name: Das Anzeigen von \gls{exif}-Daten
		\item Teilnehmende Akteure:
		\begin{itemize}
			\item	Fotograf A.: Benutzer des Programms		
		\end{itemize}
		\item Eingangsbedingung:
		\begin{itemize}
			\item	Das Programm befindet sich im Projektansichtsfenster eines Projekts.
			\item Fotograf A. hat vor, \gls{exif}-Daten eines bildes anzuschauen.
		\end{itemize}
		\item Ausgangsbedingung:
		\begin{itemize}
			\item	Es werden im Fenster \gls{exif}-Datenbereich \gls{exif}-Daten angezeigt.	
		\end{itemize}
		\item Ereignisfluss:
		\begin{itemize}
			\item Fotograf A. klickt entweder auf ein Bild im Inhaltsfenster oder auf eines in der bilderliste der Bildmenge, um dieses zu markieren.
			\item Der Bereich mit den \gls{exif}-Daten wird hierbei aktualisiert und die \gls{exif}-Daten des markierten Bildes ausgegeben.
		\end{itemize}
		\item Spezielle Anforderungen:
		\begin{itemize}
			\item	Es finden nur gültige Eingaben statt.
			\item Es befinden sich Bilder in der aktiven Bildermenge.
			\item Die \gls{exif}-Daten können angezeigt werden.			
		\end{itemize}			
	\end{itemize}
		
	\subsubsection{Auswertungsmanagement:}
Hier gilt Allgemein: Wenn Eingaben gemacht werden die nicht korrekt sind, oder nicht alle notwendigen Parameter und Werte bestimmt sind, kann entweder der Vorgang nicht fortgesetzt werden, oder im Endeffekt keine Auswertung erstellt werden. Dies wird auch durch ausgegraute Schaltflächen signalisiert. Auf kritische Stellen wird mit einer Markierung und einer Nachricht hingewiesen. Über diverse "`Vorwärts"' und "`Rückwärts"' Funktionen ist es auch gut möglich flexibel zu arbeiten und zwischen den folgenden Optionen hin und her zu springen. Einen erneuten Einstieg bietet hier der Button "`Öffnen"'.
 
\begin{description}
\item[Anwendungsfall 19]
\end{description}
 
\begin{itemize}
\item Name: Das Erstellen einer Auswertung
\item Teilnehmende Akteure:
\begin{itemize}
\item Fotograf A.: Benutzer des Programms
\end{itemize}
\item Eingangsbedingung:
\begin{itemize}
\item Das Programm befindet sich im Projektansichtsfenster eines Projekts.
\item Fotograf A. hat vor, eine Auswertung zu erstellen.
\end{itemize}
\item Ausgangsbedingung:
\begin{itemize}
\item Es wurde eine Auswertung erstellt, mit der man arbeiten kann.
\end{itemize}
\item Ereignisfluss:
\begin{itemize}
\item Fotograf A. klickt mit der Maus im Bereich Auswertung auf "`Erstellen"'.
\item Der "`Auswertung erstellen"' Wizard erscheint. Dieser ist jederzeit mit dem Button "`Schließen"' abbrechbar.
\item Hier ist zuerst ein Name für die Auswertung einzutragen und dann optional eine Beschreibung.
\item Fotograf A. gibt einen korrekten Namen ein.
\item Fotograf A. wählt mit einem Mausklick einen Diagrammtyp an.
\item Zu jedem ausgewählten Diagrammtyp wird eine kleine Vorschau inklusive Beschreibung angezeigt.
\item Beim Klick auf "`Vorwärts"' wird der zuletzt ausgewählte Diagrammtyp übernommen.
\item Nun erscheint ein neues Fenster namens: Parameter.
\item Dort trifft Fotograf A. seine gewünschten Einstellungen, welche Diagrammspezifisch angezeigt werden. 
\item Beim klick auf "`Vorwärts"' werden die getroffenen Einstellungen übernommen.
\item Hat Fotograf A. Boxplot als Diagrammtyp ausgewählt, so hat er zusätzliche Einstellungen über den \gls{wilcoxon}, welchen er zunächst einmal an und aus schalten kann.
\item Fotograf A. klickt danach wieder auf "`Vorwärts"' um zum nächsten Fenster zu gelangen.
\item Jetzt wird die Bildmengenverwaltung innerhalb einer Auswertung angezeigt.
\item Oben links werden alle Bildmengen des aktuellen Projekts angezeigt. Diese kann man über die Pfeile in der Mitte nach Rechts und nach Links schieben.
\item Wird eine Bildmenge zur Auswertung herangezogen und ist somit oben rechts, so wird das Fenster unten links mit den \gls{exif}-Keywords aktualisiert und Keywords angezeigt falls welche in der Bildmenge vorhanden sind. Duplikate werden dabei in einem Eintrag zusammengefasst.
\item Hat Fotograf A. nun alles Relevantes für seinen Diagrammtyp angegeben, so kann er auf den Button "`Ansicht"' klicken um das Diagrammansichtfenster zu öffnen.
\item Fotograf A. kann auch auf "`Übernhemen"' klicken um seine Einstellungen mit der Auswertung zu verankern.
\item Fotograf A. klickt auf "`Übernehmen"'.
\item Dem Fotograf A. wird das Projektansichtsfenster mit seiner eben erstellten Auswertung in der Auswertungsliste angezeigt.
\item 
\end{itemize}
\item Spezielle Anforderungen:
\begin{itemize}
\item Es finden nur gültige Eingaben statt.
\end{itemize}
\end{itemize}
 
\begin{description}
\item[Anwendungsfall 20]
\end{description}
 
\begin{itemize}
\item Name: Öffnen einer Auswertung
\item Teilnehmende Akteure:
\begin{itemize}
\item Fotograf A.: Benutzer des Programms
\end{itemize}
\item Eingangsbedingung:
\begin{itemize}
\item Das Programm befindet sich im Projektansichtsfenster eines Projekts.
\item Fotograf A. hat vor Änderungen an Auswertung vorzunehemen, oder sich eine Auswertung anzeigen zu lassen.
\end{itemize}
\item Ausgangsbedingung:
\begin{itemize}
\item Es werden die notwendigen Fenster zum Bearbeiten bzw. für die Ansicht geöffnet.
\end{itemize}
\item Ereignisfluss:
\begin{itemize}
\item Fotograf A. klickt mit der Maus in der Liste der Auswertungen auf die Auswertungen, die er öffnen will, um diese zu markieren.
\item Fotograf A. klickt nun auf den Button "`Öffnen"' im Auswertungsbereich um in das Auswertungsinterne Fenster zu gelangen.
\item Dies ist auch über die Auswertungsliste per Doppelklick möglich.
\item Wird nun das Auswertungsinterne Fenster angezeigt können wie bereits erwähnt Einstellungen vorgenommen werden siehe "`Anwendungsfall 19"'. Das wechseln der Fenster wird hier aber nicht mehr über "`Vorwärts"' und "`Rückwärts"' realsiert, sondern über einzelne Fenstertabs.
\item Will man zur Diagrammansicht kann man jederzeit auf "`Anzeigen"' klicken, falls dieses anwählbar ist.
\end{itemize}
\item Spezielle Anforderungen:
\begin{itemize}
\item Es finden nur gültige Eingaben statt.
\end{itemize}
\end{itemize}
 
\begin{description}
\item[Anwendungsfall 21]
\end{description}
 
\begin{itemize}
\item Name: Auswertung anzeigen 
\item Teilnehmende Akteure:
\begin{itemize}
\item Fotograf A.: Benutzer des Programms
\end{itemize}
\item Eingangsbedingung:
\begin{itemize}
\item Das Programm befindet sich im Projektansichtsfenster eines Projekts.
\item Fotograf A. hat vor, eine Auswertung anzeigen zu lassen.
\item "`Anzeigen"' ist anwählbar.
\end{itemize}
\item Ausgangsbedingung:
\begin{itemize}
\item Es wird eine Anzeige eines Diagramms ausgegeben.
\end{itemize}
\item Ereignisfluss:
\begin{itemize}
\item Fotograf A. klickt mit der Maus in der Liste der Auswertungen auf die Auswertungen, die er öffnen will, um diese zu markieren.
\item Fotograf A. klickt nun auf den Button "`Öffnen"' im Auswertungsbereich um in das Auswertungsinterne Fenster zu gelangen.
\item Dies ist auch über die Auswertungsliste per Doppelklick möglich.
\item Fotograf A. klickt nun auf "`Anzeigen"'.
\item Das Diagramm wird in einem neuen Fenster präsentiert.
\end{itemize}
\item Spezielle Anforderungen:
\begin{itemize}
\item Es finden nur gültige Eingaben statt.
\item \gls{exif}-Daten können erfolgreich gelesen werden.
\item Diagramme können erfolgreich erstellt werden.
\item Alle Einstellungen sind korrekt.
\end{itemize}
\end{itemize}
 
\begin{description}
\item[Anwendungsfall 22]
\end{description}
 
\begin{itemize}
\item Name: Sich in einer Auswertung bewegen.
\item Teilnehmende Akteure:
\begin{itemize}
\item Fotograf A.: Benutzer des Programms
\end{itemize}
\item Eingangsbedingung:
\begin{itemize}
\item Das Programm befindet sich in einer der beiden 3D Diagrammansichten.
\end{itemize}
\item Ausgangsbedingung:
\begin{itemize}
\item Fotograf A. hat sich am gewünschten Punkt mit gewünschter Sicht positioniert.
\end{itemize}
\item Ereignisfluss:
\begin{itemize}
\item Fotograf A. klickt mit der rechten Maustaste in den Diagrammbereich und hält die Maustaste gedrückt. Er kann so die Sicht frei rotieren.
\item Fotograf A. kann die Sicht mittels der mittleren Maustaste und dem Mausrad heranzoomen und wegzoomen.
\item Will man die Sicht zurücksetzen aud Default, so klickt man auf "`Ansicht zurücksetzen"'.
\item Das aktuelle Diagrammfenster wird mit jeder Eingabe aktualisiert.
\end{itemize}
\item Spezielle Anforderungen:
\begin{itemize}
\item Es finden nur gültige Eingaben statt.
\item Das Diagramm kann komplett erzeugt und angezeigt werden.
\end{itemize}
\end{itemize}
 
\begin{description}
\item[Anwendungsfall 23]
\end{description}
 
\begin{itemize}
\item Name: Eine Auswertung als Bild exportieren.
\item Teilnehmende Akteure:
\begin{itemize}
\item Fotograf A.: Benutzer des Programms
\end{itemize}
\item Eingangsbedingung:
\begin{itemize}
\item Das Programm befindet sich im Diagrammansichtsfenster eines Projekts.
\item Fotograf A. hat vor seine aktuelle Auswertungsansicht als \gls{jpg} zu exportieren.
\end{itemize}
\item Ausgangsbedingung:
\begin{itemize}
\item Es wurde eine \gls{jpg} Datei mit der aktuellen Anischt als Bild erstellt.
\end{itemize}
\item Ereignisfluss:
\begin{itemize}
\item Fotograf A. passt, wenn es Möglich ist, je nach Diagrammtyp die aktuelle Ansicht an.
\item Fotograf A. klickt auf den Button "`Als Bild exportieren"'.
\item Ein Filemanagerfenster wird aufgerufen, in welchem man ein Zielverzeichnis auswählt.
\item In einem Textfeld gibt man den gewünschten Dateinamen an.
\item Man bestätigt den Export mit "`Ok"' 
\end{itemize}
\item Spezielle Anforderungen:
\begin{itemize}
\item Es finden nur gültige Eingaben statt.
\item Die Statistik muss fehlerfrei dargestellt sein.
\item Die aktuelle Diagrammansicht muss eventuell berücksichtigt werden.
\end{itemize}
\end{itemize}
 
\begin{description}
\item[Anwendungsfall 24]
\end{description}
 
\begin{itemize}
\item Name: Diagrammansicht schließen.
\item Teilnehmende Akteure:
\begin{itemize}
\item Fotograf A.: Benutzer des Programms
\end{itemize}
\item Eingangsbedingung:
\begin{itemize}
\item Das Programm befindet sich im Diagrammansichtsfenster eines Projekts.
\item Fotograf A. hat vor das Diagrammansichtsfenster zu schließen.
\end{itemize}
\item Ausgangsbedingung:
\begin{itemize}
\item Ein Diagrammansichtsfenster wurde geschlossen.
\end{itemize}
\item Ereignisfluss:
Erste Möglichkeit:
\begin{itemize}
\item Fotograf A. klickt mit der Maus auf "Auswertung bearbeiten"'.
\item Nun wird wieder das Auswertungsinterne Fenster angezeigt.\\\\Zweite Möglichkeit:\\
\item Fotograf A. klickt mit der Maus auf "Schließen"'.
\item Das Projektansichtsfenster wird geöffnet. Alle anderen Fenster des Programms geschlossen.
\end{itemize}
\item Spezielle Anforderungen:
\begin{itemize}
\item Es finden nur gültige Eingaben statt.
\end{itemize}
\end{itemize}

\begin{description}
\item[Anwendungsfall 25]
\end{description}
 
\begin{itemize}
\item Name: Auswertung schließen.
\item Teilnehmende Akteure:
\begin{itemize}
\item Fotograf A.: Benutzer des Programms
\end{itemize}
\item Eingangsbedingung:
\begin{itemize}
\item Das Programm befindet sich in der Auswertungsinternen Ansicht.
\item Fotograf A. hat vor eine Auswertung zu schließen.
\end{itemize}
\item Ausgangsbedingung:
\begin{itemize}
\item Eine Auswertung wurde geschlossen.
\end{itemize}
\item Ereignisfluss:
\begin{itemize}
\item Fotograf A. klickt auf den Button "`Übernhemen"'.
\item Die Auswertungsinternen Ansicht schließt sich.
\item Das reine Projektansichtsfenster ist nun offen.
\end{itemize}
\item Spezielle Anforderungen:
\begin{itemize}
\item Es finden nur gültige Eingaben statt.
\end{itemize}
\end{itemize}