\subsection{Anwendungsfälle}
	\begin{itemize}
		\item Projektmanagement:
		
		Name: Erstellen eines neuen Projekts
		\begin{itemize}
			\item Teilnehmender Akteure:
			\begin{itemize}
				\item	Fotograf A.: Benutzer des Programms.
			\end{itemize}
			\item Eingangsbedingung:
			\begin{itemize}
				\item Fotograf A. besitzt das Programm.
				\item Fotograf A. will Bilder analysieren bzw. eine Auswertung erstellen.						
			\end{itemize}
			\item Ausgangsbedingung:
			\begin{itemize}
				\item	Fotograf A. hat ein Projekt, mit welchem er arbeiten kann.		
			\end{itemize}
			\item Ereignisfluss:
			\begin{itemize}
				\item Fotograf A. startet das Programm auf seinem Computer.
				\item Fotograf A. bekommt das Projektverwaltungsfenster angezeigt. Es befindet sich noch kein Projekt in der Liste.
				\item Fotograf A. klickt auf Neues Projekt erstellen.
				\item Fotograf A. gibt einen gültigen Namen für sein Projekt ein.
				\item Fotograf A. bestätigt seine Eingabe.
				\item Fotograf A. gelangt in das Hauptfenster seines Projekts und kann mit seinem Vorhaben beginnen.							
			\end{itemize}
			\item Spezielle Anforderungen:
			\begin{itemize}
				\item	Das Programm muss korrekt auf dem PC eingerichtet sein.		
			\end{itemize}			
		\end{itemize}
		
		Name: Entfernen eines Projekts
		\begin{itemize}
			\item Teilnehmender Akteure:
			\begin{itemize}
				\item	Fotograf A.: Benutzer des Programms.	
			\end{itemize}
			\item Eingangsbedingung:
			\begin{itemize}
				\item	Das Programm befindet sich im Projektverwaltungsfenster.		
			\end{itemize}
			\item Ausgangsbedingung:
			\begin{itemize}
				\item	Ein ausgewähltes Projekt wird aus dem Programm und vom Computer entfernt.		
			\end{itemize}
			\item Ereignisfluss:
			\begin{itemize}
				\item Fotograf A. klickt auf das Projekt welches er entfernen will um es zu markieren.
				\item Fotograf A. klickt auf den Button Projekt entfernen.
				\item Fotograf A.	bestätigt Sicherheitsabfrage mit Ja.				
			\end{itemize}
			\item Spezielle Anforderungen:
			\begin{itemize}
				\item	Es existiert mindestens ein Projekt.		
			\end{itemize}			
		\end{itemize}
		
		Name: Öfnnen eines Projekts
		\begin{itemize}
			\item Teilnehmender Akteure:
			\begin{itemize}
				\item	Fotograf A.: Benutzer des Programms.		
			\end{itemize}
			\item Eingangsbedingung:
			\begin{itemize}
				\item	Das Programm befindet sich im Projektverwaltungsfenster.		
			\end{itemize}
			\item Ausgangsbedingung:
			\begin{itemize}
				\item	Ein bereits erstelltes Projekt ist vollständig geladen und wird Fotograf A. angezeigt. Es kann bearbeitet werden.		
			\end{itemize}
			\item Ereignisfluss:
			\begin{itemize}
				\item Erste Möglichkeit:
				\begin{itemize}
					\item Fotograf A. klickt einmal auf das zu öffnende Projekt um es zu markieren.
					\item Fotograf A. klickt auf Button Projekt öffnen (TODO) um zum Projekt zu gelangen.
				\end{itemize}
				\item Zweite Möglichkeit:
				\begin{itemize}
					\item Fotograf A. klickt per Doppelklick direkt aud Projektnamen um es direkt zu öfnnen.					
				\end{itemize}
				\item Fotograf A. bekommt sein ausgewähltes projekt angezeigt und kann es bearbeiten.					
			\end{itemize}
			\item Spezielle Anforderungen:
			\begin{itemize}
				\item			
			\end{itemize}			
		\end{itemize}
		
		Name: Kopieren eines Projekts
		\begin{itemize}
			\item Teilnehmender Akteure:
			\begin{itemize}
				\item	Fotograf A.: Benutzer des Programms		
			\end{itemize}
			\item Eingangsbedingung:
			\begin{itemize}
				\item			
			\end{itemize}
			\item Ausgangsbedingung:
			\begin{itemize}
				\item			
			\end{itemize}
			\item Ereignisfluss:
			\begin{itemize}
				\item
				\item
				\item
				\item							
			\end{itemize}
			\item Spezielle Anforderungen:
			\begin{itemize}
				\item			
			\end{itemize}			
		\end{itemize}	
	\end{itemize}